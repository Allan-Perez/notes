\section{Week 7 - 04 Mar 2022 - Intro to topological spaces }
Note there're no lectures in week 8, due to strikes.
\begin{definition}
  Let $X,Y$ be topological spaces. A function $f:X\to Y$ is open if whenever
  $U\subset X$ is open in $X$, $f(U)\subset Y$ is open in $Y$.
  \label{<+label+>}
\end{definition}
\begin{proposition}
  Let $f:X\to Y$ be a function between topological spaces, and assume $f$ is
  bijective. Then $f^{-1}$ is continuous if an only if $f$ is open.
  \label{<+label+>}
\end{proposition}
\begin{proof}
  TODO.
\end{proof}
We now look at several examples, on which we take the topology on $\RR^n$ to be
the Euclidean topology.
\begin{proposition}
  Any open interval on $\RR$ is homeomorphic to $\RR$.
  \label{<+label+>}
\end{proposition}
\begin{proof}
  We claim the map $f:(-\frac{\pi}{2}, \frac{\pi}{2})\to\RR:x\mapsto\tan x$ is a
  homeomorphism. Since by the previous lecture any open interval is homeomorphic
  to $(-\frac{\pi}{2}, \frac{\pi}{2})$, the result follows. TODO.
\end{proof}

\begin{proposition}
  The open interval $(0,1)$ is homeomorphic to $(0,\infty)$, and $[0,1)$ is
  homemorphic to $[0,\infty)$.
  \label{<+label+>}
\end{proposition}
\begin{proof}
  The homeomorphism in both cases is given by 
  \[f(s) = \frac{s}{1-s}, \quad f^{-1}(t) =\frac{t}{1+t}.\]
  TODO.
\end{proof}
\begin{exercise}
  Finish the proofs of the previous 2 propositions, by showing that the given
  functions are indeed homeomorphisms (bijective, continuous, open).
\end{exercise}
\begin{example}
  The unit square $[-1,1]^2\subset \RR^2$ is homeomorphic to the unit disk
  $D^2\subset \RR^2$. A particular homeomorphism is given by 
  \[f(x) = \begin{cases}
      \frac{d_{\infty}(0,x)}{d_2(0,x)}x, & x\neq 0\\
      0, & x=0
    \end{cases}\]
  With inverse
  \[f^{-1}(x) = \begin{cases}
      \frac{d_2(0,x)}{d_{\infty}(0,x)}x, & x\neq 0\\
      0, & x=0
    \end{cases}\]
  TODO: Check these functions are inverses and that they are continuous
  (Exercise 4, Sheet 5 might be of use).
\end{example}
In fact the homeomorphism given above naturally applies to $\RR^n$, i.e. shows
that any cube in any dimension is homeomorphic to the sphere in that same
dimension.

\subsection{The subspace topology}
Recall that for a metric space, any subset of the space inherits the metric. The
same is true for a topological space.
\begin{definition}
  Let $X$ be a topological space, and let $A\subset X$. The subspace topology on
  $A$ is that on which if $U\subset A$ is open if and only if $U=A\cap V$ for
  some open subset $V\subset X$. In other words, if $T_X$ is the topology on
  $X$,  then the subspace topology on $A$ is 
  \[T_A = \{V\cap A : V\in T_X\}.\]
  \label{<+label+>}
\end{definition}

\begin{proposition}
  The subspace topology is a topology.
  \label{<+label+>}
\end{proposition}
\begin{proof}
  TODO.
\end{proof}

\begin{example}
  Let $\ZZ\subset\RR$ be the subset of integers in the real numbers. The
  subspace topology on $\ZZ$ coming from the usual Euclidean topology on $\RR$
  is the discrete topology. TODO: Show this is true by showing that singletons
  are open, and note how this implies the result.
\end{example}
Note that the subspace metric on $\ZZ$ is not the discrete metric (it's just
euclidean restricted to $\ZZ$) or even strongly equivalent. On the other hand,
considering the cofinite topology on $\RR$, the subspace topology on $\ZZ$ is
again the cofinite topology: $U\subset\ZZ$ is open iff $U=\emptyset$ or
$\ZZ\setminus U$ has finitely many elements.

The following generalises the fact that identity maps are continuous, and it's
the key property of the subspace topology.
\begin{lemma}
  Let $X$ be a topological space, and $A\subset X$ a subspace. The inclusion
  $i:A\to X$ is continuous (note: it's not the identity function, since the
  domain and codomain are different when considering a proper subset).
  \label{<+label+>}
\end{lemma}
\begin{proof}
  Let $U\subset X$ be open. Then $i^{-1}(U)=\left\{ a\in A: i(a)=a\in U \right\}
  = U\cap A$, which is open by definition. Hence the preimage of open sets are
  open, and the result follows.
\end{proof}
This lets us modify the domain and codomains of our function, without affecting
the property of continuity (i.e. continuity does not depend on domain or
codomain, but rather only on the assignation rule).
