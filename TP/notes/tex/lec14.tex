\section{Week 5 - 09 Feb 2022 - Open and Closed Sets}
Following from the theorem on preimages of open sets for continuous functions,
we have the following corollary,
\begin{corollary}
  Let $(X,d_X), (Y,d_Y)$ be metric spaces. Then $f:X\to Y$ is
  $(d_X,d_Y)$-continuous if and only if for each closed subset $V\subset Y$, the
  preimage $f^{-1}(V)$ is closed in $X$.
\end{corollary}
\begin{proof}
  TODO. Exercise. Just check complements, which must be open, and use the
  previous theorem.
\end{proof}
Before we can use these results to show sets are open or closed, we need a nice
set of continuous functions.
\begin{definition}
  A polynomial function $p:\RR^n\to\RR$ is a function of the form
  \[f(x_1,\cdots, x_n) = \sum_{i_1,\cdots, i_n =0}^m a_{i_1,\cdots, i_n}
  x_1^{i_1}x_2^{i_2}\cdots x_n^{i_n},\]
  For $m\in\NN$ and $a_{i_1, \cdots, i_n}\in \RR$.
  \label{<+label+>}
\end{definition}
So for example, for $n=1$, we just get the familiar 1-variable polynomial
$h(x)=x^2+\sqrt{2}x-1$, and for $n=2$ we just get polynomials like
$f(x,y)=x^2+y^2, j(x,y)=x^2y-4x^2 -xy + y^2$.

\begin{lemma}
  For each $1\leq i\leq n$, the function $f_i:\RR^n\to\RR:(x_1,\cdots,
  x_n)\mapsto x_i$ is $(d_2,d)$-continuous.
  \label{<+label+>}
\end{lemma}
\begin{proof}
  Let $\vec{a}\in\RR^2$ and $\eps>0$. Observe that for $1\leq i \leq n$, 
  \[|x_i-a_i| \leq \sqrt{(x_1-a_1)^2 + \cdots + (x_n-a_n)^2},\]
  Hence we have $d_2(\vec{x},\vec{a})< \eps \implies
  d(f_i(\vec{x}),f_i(\vec{a}))=d(x_i,a_i)<\eps$. Hence $f_1$ is continuous with
  $\delta=\eps$.
\end{proof}

\begin{proposition}
  All polynomial functions $p:\RR^n\to\RR$ are continuous (in Euclidean metric,
  and by strong equivalence also in taxicab and chessboard metrics).
  \label{<+label+>}
\end{proposition}
\begin{proof}
  Linear combinations and products of continuous real-valued functions are
  continuous, so the result follows from the above lemma,
\end{proof}

\begin{example}
  Consider the set open square $(a,b)\times(c,d)$ in $\RR^2$ with the euclidean
  metric. We show this set is indeed open.  By the above lemma, the functions
  $f(x,y)=x, g(x,y)=y$ are continuous. We showed that $(a,b),(c,d)$ are open in
  $\RR$, hence it follows that $U_1=f^{-1}( (a,b))=(a,b)\times\RR$, and similarly
  $U_2=g^{-1}( (c,d))=\RR\times (c,d)$, are open sets in $\RR^2$. Observe that
  $(a,b)\times(c,d)=U_1\cap U_2$, an intersection of two open sets, is open by
  Proposition \ref{prop:openSetsUnionIntersections}.
\end{example}
\begin{example}
  The set $A=\{(x,y)\in\RR^2 : 1<x^2+y^2<4\}$ is open in $(\RR^2, d_2)$ since
  $A= f^{-1}( (1,4))$ for $f(x,y)=x^2+y^2$, and $(1,4)$ is open and $f$ is a
  continuous function.

  The set $D=\{(x,y)\in\RR^2 : x=y+2\}$ is closed in $\RR^2$ since
  $D=g^{-1}(\{2\})$ for $g(x,y)=x-y$, and $\{2\}$ is closed and $g$ is
  continuous.
\end{example}

\begin{remark}
  Why do we use preimages and not images? We show this doesn't work with a
  couple of examples. Consider the continuous function
  $f:\RR^2\to\RR:(x,y)\mapsto 0$, but for any nonempty open set $U\subset\RR^2$
  we have $f(U)=\{0\}$, which is not open. Similarly, for the continuous
  function $g:\RR\to\RR^2:x\mapsto(x,0)$, the image $g( (a,b))$ is not open.
  However, considering the images, gives another object, called \emph{open
  functions} (whenever the image of an open set is open). Similarly, we have
  \emph{closed functions} whenever images of closed sets are closed. However,
  continuous functions need not to be open or closed, and open or closed
  functions don't need to be continuous.
  \label{<+label+>}
\end{remark}


\subsection{Open sets and distinct metrics}
Consider the metric spaces $(X,d),(X,d')$, where only the metrics are different.
Then open sets are different in each metric space. Consider the case in $\RR^2$,
where the set $U=B_{(\RR,d)}(a,r)=(a-r,a+r)$ is open in $\RR$, but $U$ is not
open in $(\RR^2,d_2)$, and it is open in $(\RR^2, D)$, where $D$ is the railway
metric.
\begin{proposition}
  Let $d,d'$ be metrics on $X$, and assume $\exists c>0$ s.t. 
  \[d(x,y)\leq cd'(x,y) \forall x,y\in X.\]
  Then if $U$ is open in $(X,d)$ is is also open in $(X,d')$.
  \label{prop:equivalenceAndOpenSets}
\end{proposition}
\begin{proof}
  TODO. Exercise. Show that any open ball in $(X,d)$ contains an open ball in
  $(X,d')$.
\end{proof}
Note that for strongly equivalent metrics, the open sets are the same.
