\section{Week 4 - 31 Jan 2022 - Cauchy Sequences and Completeness}
Recall
\begin{definition} [Convergence of sequence in a metric space]
  Let $(X,d)$ be a metric space, and let $(x_n)$ be a sequence in $X$, and let
  $a\in X$. We say the sequence $(x_n)$ converges to $a$ if for any $\eps>0$
  there exists $N\in\NN$ s.t. for any $n\geq N$, we have $d(x_n,a)<\eps$. In
  other words, $n>N\implies x_n\in B_{X}(a,\eps)$.
  \label{<+label+>}
\end{definition}
We would like to have a way of testing whether a given sequence converges
without having to come up with the limit itself. We start, as per usual, with an
example in the real numbers.

\begin{definition}[Real Cauchy sequence]
  A sequence $(x_n)$ in $\RR$ is a Cauchy sequence if 
  \[\forall\eps>0 \exists N\in\NN : n,m>N \implies d(x_m,x_n)<\eps.\]
  \label{def:cauchyRR}
\end{definition}

\begin{lemma}[Bolzano-Weirstrass Theorem]
  Let $(x_n)$ be a real sequence, and suppose there exists $L\in\RR$ s.t.
  $|x_n|\leq L$ for all $n\in\NN$. Then there exists a subsequence $(x_k)$ of
  $(x_n)$ that converges to some $a\in\RR$.
  \label{<+label+>}
\end{lemma}
\begin{proof}
  TODO.
\end{proof}<++>

\begin{theorem}
  A sequence in $\RR$ is Cauchy iff it is convergent.
\end{theorem}
\begin{proof}
  Done it in Week 1 for Complex Methods. Recall that right to left direction was
  easy (triangle inequality), but the tricky one was the forward direction
  (cauchy implies convergence). This was done by using the Bolzano-Weierstrass
  theorem and the boundedness of a Cauchy sequence.

  Let $(x_n)$ be a Cauchy sequence in $\RR$. Let $\eps=1$. Then $\exists
  N\in\NN$ s.t. $m,n>N \implies |x_m-x_n|<1$. For $m>N$,
  $|x_m|=|x_{N+1}-x_{N+1}+x_M| \leq |x_N+1| + |x_m-x_{N+1}| < |x_N+1|+1$.
  Let $R=\max (|x_1|,\cdots, |x_N|, |x_{N+1}|+1)$. Hence $|x_n|\leq R$ for any
  $n\in NN$. Hence $(x_n)$ is a bounded sequence. Bolzano-Weirstass theorem
  gives us a subsequence $(x_{n_k})$ that converges to some $c\in\RR$. Finally,
  since we have, for any $\eps>0$,
  \[\exists N\in\NN : j>N \implies |x_{n_j} - c| < \eps/2, \]
  \[\exists M\in\NN : m,n>M \implies |x_n-x_m|< \eps/2\]
  Choose $L=\max (N,M)$, so that for $m,n>L$ we have $|x_m-c|<\eps/2$ and
  $|x_m-x_n|<\eps/2$, it follows that $|x_n-c|=|x_m-c|+|x_m-x_n| < \eps$, by the
  triangle inequality. Hence the sequence $(x_n)$ converges.

\end{proof}

\begin{definition}
  A sequence $(x_n)$ in a metric space $(X,d)$ is Cauchy if for all $\eps>0$
  there exists $N\in\NN$ s.t. 
  \[m,n>N \implies d(x_m,x_n)<\eps.\]
  \label{<+label+>}
\end{definition}

\begin{proposition}
  Suppose $x_n\to a$ in a metric space $(X,d)$. Then $(x_n)$ is Cauchy.
  \label{<+label+>}
\end{proposition}
\begin{proof}
  TODO. Triangle inequality.
\end{proof}
\begin{remark}
  Note how the converse, in more general metric spaces, is not true!
\end{remark}
\begin{example}
  In $\RR$, $1/n\to 0$, hence $(1/n)$ is convergent, and by extension Cauchy.
  Now, consider $X=(0,1]\subseteq\RR$ with the subspace metric. Note that the
  sequence is in $X$ and is still Cauchy,
  \[\forall \eps>0 \exists N\in\NN : n,m>N \implies
  d_{\RR}(x_m,x_n)=d_X(x_m,x_n)<\eps\]
  However, observe that the sequence does not converge in $X$ (if it did, then
  $1/n\to a$ for $a>0$, and the same should be true when considering $X=\RR$,
  and this would contradict the uniqueness of limits).
\end{example}
Heuristically we observe that the problem of Cauchy sequences not converging in
more general metric spaces is that these allow for metric spaces that have
\emph{gaps} in the embedding space -- think of $\QQ\subset\RR$.
\begin{definition}[Complete metric space]
  A metric space $(X,d)$ is complete if every Cauchy sequence in $X$ is
  convergent in $X$.
  \label{def:completnessMetricSpace}
\end{definition}
One may ask if there's anything we can say about the relationship between
general metric spaces and complete ones.
\begin{theorem} [Non-examinable theorem]
  Let $(X,d)$ be any metric space. Then there exists a metric space
  $(\hat{X},\hat{d})$ s.t. 
  \begin{enumerate}
    \item $(\hat{X}, \hat{d})$ is complete,
    \item There exists an injective map $\phi:X\to\hat{X}$ s.t.
      \[\hat{d}(\phi(x),\phi(y))=d(x,y),\]
    \item For all $\hat{x}\in \hat{X}$ and any $\eps>0$, $B_{\hat{X}}
      (\hat{x},\eps)\cap \phi(X) \neq \emptyset$.
  \end{enumerate}
  \label{<+label+>}
\end{theorem}
\begin{proof}
  Let $X(C)$ be the set of Cauchy sequences in $X$. Define $\bar{d}:C(X)\times
  C(X) \to \RR$ by $\bar{d}( (x_n),(y_n)) = \lim_{n\to\infty} d(x_n,y_n)$. It's
  worth mentioning that this is not a metric, since we may get $0$ for different
  Cauchy sequences. Define $\hat{X} = C(X)/\sim$ for some equivalence relation
  $\sim$. TODO: Finish.
\end{proof}
