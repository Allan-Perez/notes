\section{Week 9 - 09 Mar 2022 - Connectedness, Hausdorff Spaces }
Recall that in any topological space $X$, both $X$ and $\emptyset$ are open, and
also closed. They're open by the first axiom of topological spaces, and hence
have open complement (the complement of $X$ is $\emptyset$ and viceversa), so
each one is also closed, by definition of closed sets.

If $X$ is disconnected with $X=U\cup V$ for nonempty, open and disjoint subsets
$U, V$, then $X\setminus U = V, X\setminus V= U$ are open, so $U$ and $V$ are
both open and closed. If $U\subset X$ is open and closed, then $X$ is the
disjoint union of $U$ and $V=X\setminus U$, which is open and closed. We have
thus proved the following claim.
\begin{proposition}
  A topological space $X$ is connected if and only if the only subsets of $X$
  which are both open and closed are $X$ and $\emptyset$.
  \label{prop:openClosedSubsetsConnected}
\end{proposition}

\begin{example}
  Is $\ZZ$ connected? We first need to define the topology we're working on. 
  \begin{enumerate}
    \item Discrete topology (which is the same as the subspace topology from
      $\RR$): Every subset, e.g. $\{0\}$ is open and closed, so $\ZZ$ is
      disconnected.
    \item Cofinite topology (closed sets are finite sets and $\ZZ$). In this
      topology, $\ZZ$ is connected. Assume $U\subset\ZZ$ is both open and
      closed. Then if $U\neq \emptyset$, its complement $V=\ZZ\setminus U$ must
      be closed (since $U$ is open), and by the topology $V$ must be finite.
      Since $\ZZ=U\cup V$ is infinnite, $U$ must be infinite, hence $U=\ZZ$
      since this is the only infinite closed set.
  \end{enumerate}
\end{example}
\todo{Ask Matthew why is $U$ infinite? What about $n\ZZ$?}
\todo{Ask Matthew whether the existence of a single open and closed nontrivial
subset is enough to show that the set is disconnected? Since it must be the
union, the existence of one implies the existence of more than one?}

To show that spaces are connected, which is harder, we can use help from other
toolds.
\begin{theorem}
  The open interval $(0,1)$ is connected (by using the usual topology).
  \label{<+label+>}
\end{theorem}
\begin{proof}
  Suppose $X=(0,1)=U\cup V$ for $U,V$ disjoint and nonempty. We will show $U$
  and $V$ can't be both open. Let $u\in U$ and $v\in V$. Renaming if necessary,
  we may assume $0<u<v<1$, and consider $S=\{x\in U: x< v\}$. Note $u\in S$, so
  it's nonempty. Moreover, $S$ is bounded above by $v$. Hence $S$ has a least
  upper bound, $l=\sup S$. Since $u\leq l \leq v$ we have $l\in (0,1)=U\cup V$.
  Suppose $l\in U$. If $U$ is open, $\exists r>0$ s.t. $(l-r,l+r)\subset
  U=X\setminus V$. Then, since $v\not\in U$, it means $l+r<v$, and by extension
  $l+r\in S$, a contradiction to $l=\sup S$. Hence in this case $U$ is not open. 

  On the other hand, suppose $l\in V$. If $V$ is open, then there exists $r>0$
  s.t. $(l-r,l+r)\subset V$, then $l-r$ is an upper bound for $S$, contradicting
  $l=\sup S$. Hence in this case $V$ is not open.
\end{proof}

We can exploit this fact to prove that many other spaces are connected.

\begin{proposition}
  Connectedness is preserved by continuous maps. That is, for a connected
  topological space $X$ and $f:X\to Y$ being continuous, then the image
  $f(X)\subset Y$ is connected.
  \label{<+label+>}
\end{proposition}
\begin{proof}
  We show the contrapositive, $f(X)$ disconnected implies $X$ is disconnected.
  Suppose $f(X)$ is disconnected, i.e. $f(X)=U\cup V$ with $U,V$ disjoint,
  nonempty, open. Let $\tilde{U}=f^{-1}(U)$, $\tilde{V}=f^{-1}(V)$. Then
  $\tilde{U},\tilde{V}$ are nonempty and sijoint since $U$ and $V$ are. Since
  $f$ is continuous we have $\tilde{U},\tilde{V}$ are both open. 
  Finally, $X=f^{-1}(f(X))=f^{-1}(U\cup V)=f^{-1}(U)\cup
  f^{-1}(V)=\tilde{U}\cup\tilde{V}$. Hence $X$ is also disconnected, as
  required.
\end{proof}

\begin{corollary}
  If $X\cong Y$ then $X$ is connected iff $Y$ is connected. Hence, connectedness
  is a topological property, meaning it is preserved by homeomorphisms.
\end{corollary}
\begin{corollary}
  Intervals in $\RR$ are connected. This follows from the fact that any interval
  in $\RR$ is homeomorphic to one of the intervals (open, closed) from $0$ to
  $1$.
\end{corollary}
\begin{corollary}
  The set $GL_n(\RR)$ is disconnected. This follows from the fact that
  $\det:GL_n(\RR)\to \RR\setminus\{0\}$  is continuous, and surjective. Since
  $\RR\setminus\{0\}$ is disconnected, the claim follows.
  \label{<+label+>}
\end{corollary}
