\section{Week 4 - 31 Jan 2022 - Motion in smooth axisymmetric surface}
In the last lecture we looked at an example of motion on a track, and found
useful defining a $(\xi,\eta)$-space where it was easy to analyse the system. We
now look into a more general 3D motion, in an axisymmetric surface, a.k.a.
surface of revolution. We analyse this system with cylindrical coordinates. In
this kind of system, position is given by
\[\vec{x}= r\vec{e_r} + z\vec{e_z}.\]
Where $r,\vec{e_r}, z$ are functions of time. Hence the velocity and
acceleration vectors will be rather similar to the polar coordinates case,
\[\dot{\vec{x}}= \dot{r}\vec{e_r} + r\dot{\theta}\vec{e_{\theta}} +
\dot{z}\vec{e_z}.\]
\[\ddot{\vec{x}}= (\ddot{r}- r\dot{\theta}^2)\vec{e_r} +
(2\dot{r}\dot{\theta}+ r\ddot{\theta})\vec{e_{\theta}} + \ddot{z}\vec{e_z}.\]
Since it is a surface of revolution, we have $r=f(z)$ and by the chain rule
$\dot{r}=f'(z)\dot{z}$. Using this above give
\[\dot{\vec{x}}=\dot{z}\left[ f'(z)\vec{e_r}+\vec{e_{\theta}} \right]
+r\dot{\theta}\vec{e_{\theta}}.\]
Note that a parametric description of the surface is 
\[\vec{X}(Z,\Theta) = f(Z) \vec{e_r}(\Theta) + Z\vec{e_z}.\]
Recall that $\vec{X}_Z$ and $\vec{X}_{\Theta}$ are tangent directions (from
vector calculus, I'm using subscript for partial differentiation),
\[\vec{X}_Z = f'(Z)\vec{e_r}(\Theta) + \vec{e_z},\]
\[\vec{X}_{\Theta} = f(Z)\vec{e_{\Theta}}.\]
Now, recall that velocity of the particle is only in the tangent direction (from
multivariable calculus), and so the normal to the surface is perpendicular to
velocity of the particle. For a smooth surface, the reaction force $R$ is in the
direction of the normal to the surface, and so the velocity is perpendicular to
the reaction force. Hence the reaction force does no work. Therefore, we can
ignore reaciton force in conservation of energy, and we have 
\[\frac{1}{2}mv^2 + \phi = E.\]
Plugging in the values for $v$ and $\phi$ we have 

\begin{equation}
  \frac{1}{2}m\left[ \dot{z}^2 (1+f'(z)^2) + f(z)^2\dot{\theta}^2 \right] -mgz =
  E
  \label{eqn:lec10_axisym_example_conservation}
\end{equation}

Where we used $r=f(x)$ and the fact that $m\vec{g}\cdot\vec{x}=mgz$. Next, we
consider angular momentum, where 
\[\frac{d\vec{L}}{dt} = \frac{d}{dt}(m\vec{x}\times\dot{\vec{x}}) =
\vec{x}\times\vec{F} = (r\vec{e_r} + z\vec{e_z})\times (\vec{R} - mg\vec{e_z}).\]
And consider the $\vec{e_z}$ component (details skipped, but they're not hard),
\[\frac{d}{dt}(\vec{L}\cdot
\vec{e_z})=\vec{e_z}\cdot(f(z)\vec{e_r}\times\vec{R}).\]
Hence we have that $\vec{e_z},\vec{e_r},\vec{R}$ are coplanar, since $\vec{R}$
is proportional to the normal of the surface, which depends only on
$\vec{e_z},\vec{e_r}$ ($\vec{X}_Z, \vec{X}_{\Theta}$), and so it follows that
$d/dt (\vec{L}\cdot \vec{e_z})=0$, or that the angular momentum in the
$z$-component is constant. Considering the angular momentum on its own then, we
can easily find $\vec{L}\cdot\vec{e_z}=mf^2\dot{\theta}$, and so we find
$h=f^2\dot{\theta}$. We can use this to solve the equation of conservation of
energy in Equation \ref{eqn:lec10_axisym_example_conservation}.
