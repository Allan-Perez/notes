\section{Week 2 - 19 Jan 2022 - Orbits and stability}
In the last lecture we looked at the 2-body problem by looking at the center of
mass of the system and making appropriate change of variables. We found a
solution that could take only conic-sections curves, depending on a parameter.
We will look further into this example.
\begin{example}[Geostationary Orbit]
  Calculate the height at which a satellite moves at the same angular velocity
  as the earth. 
  \begin{enumerate}
    \item Radius of the Earth: $R_E$,
    \item Find radius of ciruclar orbit: $a$,
    \item Find height above surface of Earth: $H=a-R_E$.
  \end{enumerate}
\end{example}
\begin{proof}[Solution]
  First we calculate the angular momentum per unit mass,
  $h=r^2\dot{\theta}=a^2\dot{\theta}$. Note that for geostationary orbit we
  require $\dot{\theta}=\omega_E = 2\pi$ per day. Hence $h=a^2\omega_E$. From
  the previous example we have the equation of motion is
  \[\frac{d^2u}{d\theta^2} + u = \frac{G(m+M)}{h^2}.\]
  Note that $u_{\theta\theta}=0$ since $u=1/a$ is fixed. Hence we have 
  \[ u = \frac{G(m+M)}{h^2}.\]
  Where we can approximate $m+M\approx M$, and replacing $h$ it gives 
  \[a^3=\frac{G M}{h^2}.\]
  Next we try to get rid of $G$ in place of $g$, which we know to be (at the
  surface of the earth),
  \[g=\frac{GM}{R_E^2}.\]
  Hence it follows that 
  \[a^3 = \frac{R_E^2 g}{\omega_E^2}\]
  \[\implies H = \frac{R_E^2 g}{\omega_E^2} - R_E \approx 36000 km.\]
  By approximating $R_E, g, \omega_E$.
\end{proof}

\begin{example}
  Consider a satellite moving around a mass $M$ (say, a planet), with speed
  \[v_0 =\sqrt{\frac{\gamma}{a}}, \gamma=GM,\]
  At an angle $\alpha=\frac{\pi}{2}$ to the radius (this problem is in $\RR^2$).
  Assume $m<<M$, where $m$ is the mass of the satellite itself. Find the path of
  the orbit and determine if it is bounded. In the case it is bounded, determine
  the period of the orbit.  Initially the satellite is at a distance $a$ from
  the origin, the cm of $M$.
\end{example}
\begin{proof}[Solution]
  First we aim to find $h=r^2\dot{\theta}$. Note that $r\dot{\theta}$ is the
  $\theta$-compontent of the velocity:
  $\dot{\vec{x}}=\dot{r}\vec{e_r}+r\dot{\theta}\vec{e_{\theta}}$. Note we have
  (at $t=0$),
  \[\dot{r}= |\vec{v}\cdot\vec{e_r}|= v_o\cos\frac{\pi}{4}=
  \sqrt{\frac{\gamma}{2a}}.\]
  Similarly, 
  \[r\dot{\theta} = |\vec{v}\cdot\vec{e_{\theta}}| = v_o
  \cos\frac{\pi}{4} = \sqrt{\frac{\gamma}{2a}}.\]
  Hence, at $t=0$, we have $h=\sqrt{\frac{a\gamma}{2}}$. We can then use the
  equation of motion (approximating $m+M\approx M$), and get 
  \[\frac{d^2u}{d\theta^2} + u = \frac{GM}{h^2} = \frac{2}{a}.\]
  For intial conditions of position, take $u=\frac{1}{a}$ and
  $\theta=\theta_0=0$ (WLOG). From the results in Exercise
  \ref{exc:transformationCentralForceField}, we can use
  $\dot{r}=-h\frac{du}{d\theta}$. Hence substituting we have 
  \[\frac{du}{d\theta} = \frac{-1}{a}\]
  Which is another initial condition. Solving the ODE above, we get
  $u=\frac{2}{a} + A\cos\theta + B\sin\theta$, with ICs at $\theta=0$ of $u=1/a$
  and $u_{\theta}=-1/a$. Plugging in ICs, we get $A=-1/a$ and $B=-1/a$. Hence
  the equation of path is 
  \[u=\frac{1}{a}\left( 2-\cos\theta - \sin\theta \right)=\frac{2}{a}\left(
  1+e\cos\left( \theta-\theta_0 \right) \right)\]
  For $e=\frac{1}{\sqrt{2}}$ and $\theta_0=\frac{3\pi}{4}$. Since the
  eccentricity is between $0$ and $1$, it follows that the path is an ellipse,
  hence bounded. The next question to answer is the period of this orbit. Recall
  that $T=\int_0^T dt$ which by using the appropriate change of variables we
  find
  \[T=\int_{0}^{2\pi}\frac{dt}{d\theta}d\theta =
  \int_0^{2\pi}\frac{1}{\dot{\theta}} d\theta = \int \frac{1}{hu^2}d\theta.\]
  In order to proceed we can simply substitute by the solution $u$ to find
  (recall that $2h^2/\gamma = a$),
  \[T= \frac{h^3}{\gamma^2} \int_0^{2\pi}
  \frac{1}{(1+e\cos(\theta-\theta_0))^2}d\theta\]
  The solution to this integral is left as an exercise to the reader. The final
  answer is 
  \[T= \frac{h^3}{\gamma^2} \frac{2\pi}{(1-e^2)^{3/2}}\]
  Note that this diverges as $e\to 1$, since parabolas do not have a period.
\end{proof}
\begin{remark}
  Recall that for $A\cos\theta+B\sin\theta=e\cos(\theta-\theta_0)$.
\end{remark}


\subsection{Stability of circular orbits}
Under what conditions can a ciruclar orbit exist? note that this requires $r$ to
be a constant, so $\dot{r}=0$. Hence the equation of motion would reduce to 
\[\ddot{r}-\frac{h^2}{r^3}= \frac{F(r)}{m} \]
\[\implies r^3 = - \frac{mh^2}{F(r)}\]
So circular orbits exist only when this relation is satisfied. Moreover, note
that no circular orbits can exist if $F(r)>0$, a repulsive central force.
However, if we assume $F(r)<0$, and the solution gives $r=a$, some constant. How
do we know if the solution is stable? Write a small perturbation, and see what
happens,
\[r(t) = a + \eps(t)\]
For some $\eps<<1$ initially. If $\eps$ grows, we say the orbit is unstable. If
it decays, we say it's stable. Substituting this into the equation of motion
gives an ODE in terms of $\eps$,
\[\ddot{\eps} - \frac{h^2}{(a+\eps)^3} = \frac{1}{m}F(a+\eps)\]
Note that we can simplify the negative term in the LHS, as follows
\[\frac{-h^2}{(a+\eps)^3} = \frac{-h^2/a^3}{(1+\eps/a)^3}\]
We can expand the binomial (with negative exponent) and get,
\[\frac{-h^2/a^3}{(1+\eps/a)^3}\approx
\frac{-h^2}{a^3}+\frac{3h^2}{a^4}+O(\eps^2).\]
And we can do similarly for $F(a+\eps)$,
\[F(a+\eps) = F(a) +\eps F'(a)+ O(\eps^2)\]
And so after equating and simplifying, we get an ODE for $\eps$, 
\[\ddot{\eps} -\frac{1}{ma}(3F(a)+aF'(a))\eps=0\]
Solving for this equation, we can set $p^2=\frac{1}{ma}(3F(a)+aF'(a))$, in which
case we get an equation $\ddot{\eps}-p^2\eps =0$, which gives an exponential
growth.However, if we set $\omega^2=-\frac{1}{ma}(3F(a)+aF'(a))$, we get
$\ddot{\eps}+\omega^2\eps=0$, which gives solution $y=A\cos\omega t +
B\sin\omega t$. For $3F(a)+aF'(a)>0$ we get an unstable solution (spiriling
outwards), but for $3F(a)+aF'(a)<0$, we get an stable solution (oscillating
around the original orbit). When $3F(a)+aF'(a)=0$, the linear theory is not
enough to answer the question.
