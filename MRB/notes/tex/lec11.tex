\section{Week 4 - 04 Feb 2022 - Another example axisymmetric surface - Intro to
rotating reference frame}
\begin{example}[Particle in sphere]
  Derive the equations of motion for a particle moving inside the surface of a
  sphere of radius $a$ about the origin.
\end{example}
\begin{proof}[Solution]
  The sphere can be describe in cylindrical coordinates by
  \[r^2 + z^2 = a^2 \implies r=f(z)= \sqrt{a^2-z^2}\]
  \[f'(z)= \frac{-z}{\sqrt{a^2-z^2}}.\]
  Suppose that the particle starts its motion with height $z=-d$, in the souther
  hemisphere of the sphere where the geometry is convex -- like a bowl.
  Furthermore, suppose the partcile starts moving with speed $v_0$, parallel to
  the plane $z=0$ along the surface. Hence,
  \[\dot{\vec{x}} = \dot{z}(\vec{e_z} + f'(z)\vec{e_r}) +
  r\dot{\theta}\vec{e_{\theta}},\]
  Since $\dot{r}=\dot{r}f'(z)$ by the chain rule. By the ICs, $\dot{z}=0$,
  hence at $t=0$,
  \[\dot{\vec{x}}=r\dot{\theta} \vec{e_{\theta}}.\]
  Moreover, $v_0=|r\dot{\theta}|$. This implies that $h=r^2\theta=r(t=0) v_0$,
  where $r(t=0)=f(z(t=0))=f(-d)$ by initial conditions, so $h=f(-d)v_0$.
  Therefore, since we're in a conservative force field, 
  \[E=-mgd + \frac{1}{2}mv_0^2 = -mgd + \frac{1}{2}m\frac{h^2}{f(-d)^2}.\]
  Note that this is yet another example of an axisymmetric smooth surface, so we
  can equate with Equation \ref{eqn:lec10_axisym_example_conservation}, to get 
  \[
    \frac{1}{2}m\left[ \dot{z}^2 (1+f'(z)^2) + \frac{h^2}{f^2}\right] +mgz =
  -mgd + \frac{1}{2}m\frac{h^2}{f(-d)^2}
  \]
  This differential equation in $z$ will gives us the behaviour of $r$ since
  $r=f(z)$, and at the same time we get the behaviour of $\theta$ since
  $\dot{\theta}=\frac{h}{r^2}$.
\end{proof}

A question we may ask with the above results is the following
\begin{example}
  In the above system, what are the maximum and minimum heights achieved by the
  ball, with the given ICs.
\end{example}
\begin{proof}[Solution]
  Note that the requirement implies $\dot{z}=0$. In order to find the position
  at which these occur, we solve the differential equation in the previous
  example, and get
  \[
    \frac{1}{2}m\left[ 0 (1+f'(z)^2) + \frac{h^2}{f(z)^2}\right] +mgz =
  -mgd + \frac{1}{2}m\frac{h^2}{f(-d)^2}
  \]
  \[\implies \frac{v_0^2f(-d)^2}{f(z)^2}  +2gz = -2gd + v_0^2\]
  By plugging in the values of $f$,
  \[\implies v_0^2\left( \frac{f(-d)^2}{f(z)^2}-1 \right)+2g(z+d) =0\]
  \[\implies \frac{v_0^2}{2g}\left( \frac{a^2-d^2}{a^2-z^2}-1 \right)+z+d=0\]
  \[\implies \frac{v_0^2}{2g}\left( \frac{z^2-d^2}{a^2-z^2} \right)+z+d=0\]
  \[\implies \frac{v_0^2}{2g}(z^2-d^2)+ (z+d)(a^2-z^2)=0\]
  \[\therefore (z+d)\left( \frac{v_0^2}{2g}(z-d) + a^2-z^2  \right)=0.\]
  Which gives a solution $z=-d$, which is precisely the initial conditions. We
  can then solve the quadratic to get the other two roots. Then it's obvious
  that the maxima will be the positive discriminant, and the minima will be the
  negative discriminant.
\end{proof}

\subsection{Motion in rotating reference frames}
Consider an inertial reference frame $S$ with basis vectors
$\vec{e_1},\vec{e_2},\vec{e_3}$ , and another reference frame $S'$ with basis
vectors $\vec{e_1}',\vec{e_2}',\vec{e_3}'$ which is obtained by rotating $S$
under time-dependent transformation $R(t)$. Consider a vector $\vec{e}$ fixed in
$S'$, so that $\vec{r}\cdot\vec{e_i}=k_i$ for all $i=1,2,3$. What is
$\frac{d\vec{r}}{dt}$? We know
\[\vec{r}= \sum (\vec{r}\cdot\vec{e_i}') \vec{e_i}'\]
Hence it follows that 
\[\dot{\vec{r}}= \sum (\vec{r}\cdot\vec{e_i}') \dot{\vec{e_i}}'.\]
Note that since $\vec{r}\cdot\vec{e_i}'=k_i$, then $\dot{\vec{r}}\cdot\vec{e_i}' +
\vec{r}\cdot \dot{\vec{e_i}}'=0$ by applying the product rule. Hence (careful,
  this may not look like it makes much sense, but it does: you have to recall
that $\vec{r}$ is constant in time only wrt $S'$),
\[\dot{\vec{r}}=\sum (\dot{\vec{r}}\cdot \vec{e_i}')\vec{e_i}' = \sum
(-\vec{r}\cdot\dot{\vec{e_i}}')\vec{e_i}'. \]
Hence we can add both expressions for $\dot{\vec{r}}$ to get 
\[2\dot{\vec{r}} = \sum (\vec{r}\cdot\vec{e_i}') \dot{\vec{e_i}}' -
(\vec{r}\cdot\dot{\vec{e_i}}')\vec{e_i}'.\]
Recall that $\vec{a}\times(\vec{b}\times\vec{c})= (\vec{a}\cdot\vec{c})\vec{b} -
(\vec{a}\cdot\vec{b})\vec{c}$, hence we observe that 
\[2\dot{\vec{r}}= \sum \vec{r}\times(\dot{\vec{e_i}}'\times\vec{e_i}')\]
\[\therefore \dot{\vec{r}}=\left[\vec{r}\times  \frac{1}{2}\sum
(\dot{\vec{e_i}}'\times\vec{e_i}')\right]\]
\[= \left[  \frac{1}{2}\sum
(\vec{e_i}'\times\dot{\vec{e_i}}' )\right] \times \vec{r}.\]
Where we define $\vec{\omega}=\frac{1}{2}\sum (\vec{e_i}'\times\dot{\vec{e_i}}'
)$.
\begin{theorem}[Rotating axes theorem]
  Let $S$ be a rotating reference frame with angular velocity $\vec{\omega}$,
  and let $\vec{r}$ be a vector that is constant in $S$. Then 
  \[\dot{\vec{r}} = \omega\times\vec{r}.\]
  \label{thm:rotatingAxes}
\end{theorem}

For a vector $\vec{R}=\vec{R}(t)$ that is not constant in $S$, a rotating
reference frame, we can apply Theorem \ref{thm:rotatingAxes} to its basis, and
use the product rule (the derivation is not hard) to get
\[\dot{\vec{R}} = \dot{\vec{R}}' + \omega\times\vec{R}.\]
Where $\dot{\vec{R}}'$ is the derivative in the rotating reference frame, and
$\dot{\vec{R}}$ is the derivative in the inertial reference frame.
