\section{Week 3 - 26 Jan 2022 - Energy and work}
\begin{proposition}
  Consider motion under the only influence of a central force. Then there's no
  work being done along $\vec{e_{\theta}}$.
  \label{prop:eThetaWorkInCentralField}
\end{proposition}
\begin{proof}
  Consider motion described by a curve $C_1$ s.t. it has only component
  $\vec{e_{\theta}}$, under the influence of a central force (so
  $\vec{F}=F(r)\vec{e_r}$). Since $\vec{\dot{x}}$ has only component
  $\vec{e_{\theta}}$, it follows that $\vec{F}\cdot
  \vec{\dot{x}}=F(r)\vec{e_r}\cdot v\vec{e_{\theta}}=0$.
\end{proof}

\begin{example}
  Find the potential by considering some path from $\vec{x_0}$ to $\vec{x}$,
  under the influence of a central force.
\end{example}
\begin{proof}[Solution]
  Consider some path $C_{x_0\to x}$ s.t. it consists of two paths $C_1, C_2$
  s.t. $C_1$ has only component $\vec{e_{\theta}}$ and $C_2$ has only component
  $\vec{e_r}$. Then note that
  \[\phi =-\int_{C_{x_0\to x}} \vec{F}(\vec{y})\cdot\vec{y},\]
  \[ = -\int_{C_1}\vec{F}(\vec{y})\cdot\vec{y} -
  \int_{C_2}\vec{F}(\vec{y})\cdot\vec{y}\]
  \[= 0-\int_{C_2}\vec{F}(\vec{y})\cdot\vec{y}\]
  \[\therefore \phi = -\int_{C_2}\vec{F}(\vec{y})\cdot\vec{y},\]
  Since $d\vec{y}$ is perpendicular to $\vec{F}=F(r)\vec{e_r}$ along $C_1$. Then
  note that we can parametrise $C_2$ easily (only has component $\vec{e_r}$).
  Consider $\vec{y}=s\vec{e_r}$ where $\vec{e_r}=\frac{\vec{x}}{|\vec{x}|}$, for
  $s\in [|\vec{x_0}|,|\vec{x}|]$. Note that we can set these bound because along
  $C_1$, the quantity $|\vec{y}|$ has stayed constant, hence at the point where
  $C_2$ starts, $|\vec{y}|=|\vec{x_0}|$. Hence note that we can write 
  \[\phi = -\int_{|\vec{x_0}|}^{|\vec{x}|} F(s)\vec{e_r}\cdot\vec{e_r}ds\]
  \[\therefore \phi= -\int_{|\vec{x_0}|}^{|\vec{x}|} F(s)ds.\]
\end{proof}
\begin{remark}
  Note that when we define the potential, the $0$-energy point is really
  arbitrary, so we can set $\vec{x_0}$ wherever we want. What is important is
  the gradient.
\end{remark}

\begin{example}
  Find the potential for the inverse square law
  \[\vec{F} = -\frac{GMm}{|\vec{x}|^2} \frac{\vec{x}}{|\vec{x}|}.\]
\end{example}
\begin{proof}[Solution]
  Note that $F(r)=-\frac{GMm}{r^2}$ and let us define $\vec{x_0}$ s.t.
  $|\vec{x_0}|=\infty$. Hence from the above example, we have
  \[ \phi=- \int_{-\infty}^{|\vec{x}|}-\frac{GMm}{s^2}ds =
  -\frac{GMm}{|\vec{x}|} = -\frac{GMm}{r}.\]
\end{proof}

\begin{example}[Escape velocity]
  Find the minimal velocity required to project an object from Earth to space,
  so that it escapes Earth's gravitational field (i.e. does not fall back down).
  Assume that Earth is spherical, that the only force acting is gravity, and
  neglect Coriolis force.
\end{example}
\begin{proof}[Solution]
  Since gravity is a central force, we already know that it is conservative.
  Hence conservation of energy applies. Consider energy at the moment of launch,
  \[E=\frac{1}{2}mv_0^2 - \frac{GMm}{R},\]
  Where $R$ is the radius of Earth. As $r\to \infty$, since we're assuming that
  it gets out of the gravitational field, we have
  $E=\frac{1}{2}mv_{\infty}^2-0\geq 0$, so
  \[0\leq \frac{1}{2}mv_0^2 - \frac{GMm}{R} \implies v_0\geq
  \sqrt{\frac{2GM}{R}}= \sqrt{2gR}\approx 11 km/s.\]
\end{proof}

\begin{example}
  Consider a body approaching Earth from outer space. What is the distance of
  closest approach to the Earth's surface?
\end{example}
\begin{proof}
  By conservation of energy, the energy at the point of closest approach will be
  equal to the initial energy the body has when approaching Earth,
  \[\frac{1}{2}mv^2 - \frac{GMm}{r} = \frac{1}{2}mv_0^2\]
  \[\implies v^2 - \frac{2GM}{r} = v_0^2.\]
  Recall that in polar coordinates, $v^2 = \dot{r}^2 + r^2 \dot{\theta}^2$, and
  since the field is conservative, it follows that angular momentum is conserved
  and so $h=r^2\dot{\theta}$ is a constant, hence
  $\dot{\theta}=\frac{h}{r^2}$. Therefore the equation of energy reduces to 
  \[\dot{r}^2 + \frac{h^2}{r^2} - \frac{2GM}{r} = v_0^2.\]
  At the closest approach, $r=r_{min}$, so $\dot{r}=0$, which reduces the above
  equation into
  \[\frac{h^2}{r_{min}^2} - \frac{2GM}{r_{min}}=v_0^2.\]
  To find out $h$, we use initial conditions. Recall that angular momentum is
  conserved in this framework, so that
  $\vec{L}=m\vec{x}\times\dot{\vec{x}}=m(r\vec{e_r})\times
  (\dot{r}\vec{e_r}+r\dot{\theta}\vec{e_{\theta}})=mr^2\dot{\theta}\vec{e_z}$,
  and hence recall that $h=\frac{|L|}{m}$, but also note that
  $|L|=m|\vec{x}||\dot{\vec{x}}|\sin \theta=mrv\sin\theta$, where $\theta$ is
  the angle between the initial velocity vector and the initial position vector,
  and so $r\sin\theta=d$ is the distance between Earth and the line which has
  $\dot{\vec{x}}$ as director vector (i.e. the closest distance between Earth
    and the body if there was no gravity, that is, if the body kept travelling
  in a straight line). Hence $h=v_0 d$. Pluggin in the equation in terms of
  $r_{min}$ and solving the quadratic gives
  \[r_{min}=\frac{-GM}{v_0^2} \pm \sqrt{d^2 + (\frac{GM}{v_0^2})^2}.\]
  The only physically relevant is the positive root (otherwise we just have a
  negative distance -- \emph{What does this mean? It doesn't reach the
  planet? Hyperbolic orbit? Has something to do with the curvature experienced by
  the body, s.t. it doesn't get attracted to Earth but diverges?}).
  \todo{Ask this in the tutorial or office hour}.
  Note that to avoid hitting Earth we require $r_min> R$ where $R$ is Earth's
  radius.
\end{proof}
