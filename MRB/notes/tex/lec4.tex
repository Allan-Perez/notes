\section{Week 2 - 17 Jan 2022 - Central force field change of coordinates }
We will simplify previous conservation of energy equations, so that they are
easier to solve.
\begin{exercise}
  Simplify the equations of motion for a point mass in a central force field in
  polar coordinates, as defined by 
  \[\ddot{r} - \frac{h^2}{r^3} = \frac{F(r)}{m}\]
  \[r^2\dot{\theta} = h\in\RR\]
  \label{exc:transformationCentralForceField}
\end{exercise}
\begin{proof}[Solution]
  We will define a transformation of the variables $r,\theta$. First, observe
  that $\dot{\theta}$ has the same sign as $h$. If $\dot{\theta}>0$, then
  $\theta$ is increasing, and by extension $\theta(t)$ is invertible: we can
  define $t=t(\theta)$, so that we can have $\theta$ to be the independent
  variable. Let us define the transformation $u=\frac{1}{r(t(\theta))}$. From
  the second equation we have $\dot{\theta}= \frac{h}{r^2}= hu^2$, and by the
  inverse function theorem we get
  \[\frac{dt}{d\theta} = \frac{1}{\dot{\theta}}= \frac{1}{hu^2}\]
  Next, we aim to get $\ddot{r}$ in terms of $u$, so we differentiate $r$ wrt
  $\theta$,
  \[\frac{dr}{d\theta} = \frac{dr}{dt} \frac{dt}{d\theta} =
  \frac{1}{hu^2}\dot{r}\]
  Next, we use our substitution to get 
  \[\frac{d}{d\theta}(\frac{1}{u}) = \frac{-1}{u^2}\frac{du}{d\theta}\]
  Hence,
  \[\frac{du}{d\theta} = -\frac{1}{h}\dot{r}\]
  \[\implies \dot{r}= -h\frac{du}{d\theta}\]
  This expression is useful for ICs in our new variables. We then get $\ddot{r}$
  by differentiating again wrt $\theta$,
  \[\frac{d\dot{r}}{dt}\frac{dt}{d\theta} = -h\frac{d^2u}{d\theta^2}\]
  \[\implies \ddot{r} = -h^2u^2 \frac{d^2u}{d\theta^2}.\]
  We then use this result to get the original equation of motion into our new
  variables,
  \[-h^2u^2 \frac{d^2u}{d\theta^2}- h^2u^3 =
  \frac{F(\frac{1}{u})}{m}\]
  \[\therefore \frac{d^2u}{d\theta^2} + u = \frac{-1}{mh^2u^2}
  F(\frac{1}{u}).\]
  This is called the equation of path, where the LHS has the form of an
  oscillating system. We get the second equation to be $\frac{dt}{d\theta}=
  \frac{1}{hu^2}$. The ICs at $t=0$ corresponds specifying $u,u_{\theta}$ at
  some $\theta_0$.
\end{proof}
We will apply this result to some physical problems involving gravity. Before
that, we will introduce Newton's law of gravitation and its generalisation to
bodies with some distribution of density.
\begin{theorem}[Newton's Law of Gravitation]
  Suppose we have a point mass $m_1$ at position $\vec{x_1}$, and another point
  mass $m_2$ at position $\vec{x_2}$. Let $\vec{F_{12}}$ be the force exterted
  on $m_1$ by $m_2$. Let $\vec{F_{21}}$ be the force exterted on $m_2$ by $m_1$.
  By N3, we have $\vec{F_{21}}=-\vec{F_{12}}$. Then we have,
  \[\vec{F_{12}} = -Gm_1m_2\frac{\vec{x_1}-\vec{x_2}}{|\vec{x_1}-\vec{x_2}|^3}\]
  \label{thm:newtonGrav}
\end{theorem}
In order to generalise to density bodies, assume that $m_1$ is a point mass and
$m_2$ is a density distribution $\rho(\vec{x})$, a body $B$ of some volume
$V$,so that in the above framework, a point mass of the second body would be
$\rho(\vec{x})dV_x$. Hence the force on $m_1$ by a point mass of $m_2$ at
position $\vec{x_3}$ is 
\[F_{1m_2\vec{x_3}} = -Gm_1
\frac{\rho(\vec{x_3})dV_{x_3}(\vec{x_1}-\vec{x_3})}{|\vec{x_1}-\vec{x_3}|^3}\]
And by extension, the force exerted by $B$ will be the integration over all
point-masses of $B$, 
\[F_{12} = -Gm_1\int\int\int_{V}
\frac{\rho(\vec{x})(\vec{x_1}-\vec{x})}{|\vec{x_1}-\vec{x}|^3} dV_{x} \]
Which is a volume integration.

\begin{theorem}[Newton's Shell Theorem]
  Let $B$ be a spherically symmetric volume mass distribution, i.e. the mass of
  the $B$ depends only on the distance form the center. Then we have
  \[\vec{F_{1B}} = -Gm_1 M \frac{\vec{x_1}-\vec{x}}{|\vec{x_1}-\vec{x}|^3}\]
  Where $M$ is the total mass of $B$, $\vec{x}$ is the position of its center of
  mass, i.e. center of the sphere (since it's spherically symmetric).
  \label{thm:newtonShellThm}
\end{theorem}
This is a useful result since we can effectively replace a solid body (in the
special case of a symmetric sphere) by a point-mass placed at the center.
\begin{example}
  Consider two masses moving under gravity,
  \[m_1\ddot{\vec{x_1}} = \vec{F_{12}}, \quad m_2\ddot{\vec{x_2}} =
  \vec{F_{21}}.\]
  By Newton's third law, we have $\vec{F_{12}} = -\vec{F_{21}}$, so this implies
  $m_1\ddot{\vec{x_1}} + m_2\ddot{\vec{x_2}}=0$. We define the center of mass to
  be the positon vector,
  \[\vec{x}= \frac{m_1\vec{x_1} + m_2\vec{x_2}}{m_1+m_2}\]
  This definition implies that $(m_1+m_2)\ddot{\vec{x}} = 0$, i.e. the center of mass
  moves without acceleration -- experiences no force. WLOG assume
  $\vec{x}=\vec{0}$. We now aim to find the equation of motion. Take
  $\ddot{\vec{x_1}}= \frac{\vec{F_{12}}}{m_1}, \ddot{\vec{x_2}}=
  \frac{-\vec{F_{12}}}{m_2}$ (by N3), and so we have 
  \[\ddot{\vec{x_1}}-\ddot{\vec{x_2}} = \vec{F_{12}} (\frac{1}{m_1}+\frac{1}{m_2})\]
  Define $\vec{R}= \vec{x_1}-\vec{x_2}$, the relative position of $x_1$ wrt
  $x_2$. Moreover, define $M=\frac{m_1m_2}{m_1+m_2}$ (called \emph{reduced
  mass}), and note that we have simplified the equation to ( RHS in the form of
  central force).
  \[M\ddot{\vec{R}} = \vec{F_{12}} = F_{12}(|\vec{R}|) \frac{\vec{R}}{|\vec{R}|}\] 
  Which is a new problem. Note that $F_{12}(|\vec{R}|)=
  \frac{-Gm_1m_2}{|\vec{R}|^2}$. Suppose that $m_2>>m_1$, so that $M\approx
  m_1$. The equation of motion in this case reduces to 
  \[m_1\ddot{\vec{R}} = \frac{-Gm_1m_2}{|\vec{R}|^2} \frac{\vec{R}}{|\vec{R}|}.\]
  Using the equations of motion in terms of $u$ and $\theta$ (which we can do
  because we already proved motion is in a plane since angular momentum is
  conserved), we get
  \[u_{\theta\theta}+u = \frac{-1}{mh^2u^2} F(\frac{1}{u}) =
  \frac{Gm_2}{h^2}\]
  And the RHS here is a constant. This can be solved using familiar methods, to
  get 
  \[u=\frac{Gm_2}{h^2} + A\cos\theta + B\sin\theta\]
  \[u = \frac{Gm_2}{h^2} + C\cos(\theta-\theta_0)\]
  \[u= \frac{Gm_2}{h^2}\left( 1+e\cos(\theta-\theta_0) \right)\]
  Where $e$ is called \emph{eccentricity}. Different value of $e$ give different
  shapes, generally categorised by $e=0, e=1, e\in(0,1), e>1$ to give orbits of
  shape circle, parabola, ellipse, and hyperbola, respectively.
\end{example}
Questions I had at the end of the lecture:
- How do we transition from $\vec{R}$ to $u,\theta$? The equations in $\vec{R}$
do not have the same form as in Exercise
\ref{exc:transformationCentralForceField}.
