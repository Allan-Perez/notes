\section{Lecture 12 - 1 Mar 2022 - Liouville's Theorem, FT Algebra}
The case when the domain of $f$ is defined on $\CC$, something special happens.
\begin{theorem}[Liouville's Theorem]
  Let $f:\CC\to\CC$ be a holomorphic function and not constant. Then $f$ is
  unbounded.
  \label{<+label+>}
\end{theorem}
\begin{proof}
  Suppose $f$ is holomorphic and bounded. That is, there exists $M>0$ such that
  $|f(z)|\leq M$ for all $z\in\CC$. We now show that $f'(z)=0$ for all $z$,
  hence $f$ must be constant. By Cauchy's integral formula, we have, for some
  circle path $\gamma$ with center $z$ and radius $r>0$,
  \[f'(z)=\frac{1}{2\pi i} \oint_{\gamma}\frac{f(\zeta)}{\zeta-z}d\zeta.\]
  According to the bounds for integrals, we have 
  \[|f'(z)|\leq \frac{1}{2\pi} \frac{M}{r^2} 2\pi r = \frac{M}{r} \forall r>0. \]
  Hence $f$ must have $f'=0$, as required.
\end{proof}

\begin{theorem}[Fundamental Theorem of Algebra]
  Let $p(z)$ be a polynomial of degree $d>0$. Then $p(z)$  has at least one
  root.
  \label{<+label+>}
\end{theorem}
\begin{proof}
  Suppose that $p(z)=0$ has no solutions. Then we can define
  $f:\CC\to\CC:z\mapsto \frac{1}{p(z)}$, a holomorphic function (since $p$ is
  certainly holomorphic, and by assumption non-zero, and $1/z$ is a holomorphic
  function for the given domain, and compositions of holomorphic functions are
  holomorphic). Moreover, note that $f(z)\to 0$ as $z\to\infty$ since $d>0$, by
  definition of polynomials. This implies that there exists $K,L>0$ constants
  s.t. $|f(z)|\leq K$ for $|z|>L$. Let $\gamma$ be a circle with center $z$ and
  radius $R\geq |z|+L$. Then by Cauchy's integral formula,
  \[|f'(z)|= \frac{1}{2\pi} \left|
  \oint_{\gamma}\frac{f(\zeta)}{\zeta-z}d\zeta\right| \frac{1}{2\pi}
  \frac{K}{R^2} 2\pi R = \frac{K}{R} \forall R\geq |z|+L,\]
  Hence it follows $f'(z)=0$, a contraditiction, since $d>0$. Hence $f$ can't be
  holomorphic, i.e. $p(z)=0$ must have a solution.
\end{proof}
The proof for $p$ having $d$ roots follows immediately, since, given that we
have surely one root, we can factor out the root, and we're left with a new
polynomial of degree $d-1$, which will again surely have one root, and again can
be factorised to get a new polynomial of degree $d-2$. We carry on inductively
and it must follow that $p$ can be factorised into $d$ factors, i.e. have $d$
roots.
