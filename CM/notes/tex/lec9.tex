\section{Lecture 9 - 22 Feb 2022 - Cauchy's theorem}
Cauchy's Theorem will show us that $\oint_{\gamma} f(z)dz=0$ for any closed path
$\gamma$ in $\Omega$, provided $f:\Omega\to\CC$ is holomorphic.
\begin{definition}
  A path $\gamma:[a,b]\to\Omega\subset\CC$ is simple if and only if it doesn't
  cross with itself, with the possible exception of the endpoints (a closed
  path).
  \label{def:simplePath}
\end{definition}
\begin{theorem}
  Let $f:\omega\to\CC$ be holomorphic and $\gamma:[a,b]\to\Omega$ be a simple
  closed path such that all point inside of $\gamma$ are in $\Omega$ (i.e. no
  holes). Then,
  \[\oint_{\gamma}f(z)dz =0.\]
  \label{thm:CauchyIntegration}
\end{theorem}
To prove this theorem, we will prove smaller claims and then connect them to
give the final proof. First we consider contours which are images of rectangles.
\todo{Why do we need $\Gamma$? Isn't $\gamma$ (the image of the boundary
under a continuous deformation) enough?}
\begin{theorem}
  Let $\Omega\subset\CC$ be open and let $f:\Omega\to\CC$ be holomorphic. Let
  $R=[0,1]\times[a,b]$ be a rectangle in $\RR^2$ with boundary $\partial R$ and
  $\gamma: \partial R\to\Omega$ be a path s.t. $\gamma=\Gamma|\partial R$
  (restriction of domain to $\partial R$) for some continuous function $\Gamma:
  R\to\Omega$. Then $\oint_{\gamma}f(z)dz=0$.
  \label{thm:CauchyIntegrationRectangles}
\end{theorem}
\begin{proof}
  Let us define the function $I(R)=\int_{\gamma(R)}f(z)dz$ as the integral along
  the path $\gamma(R)=\Gamma|\partial R$, i.e. the integral along the boundary
  of the image (by $\Gamma$) of the rectangle. We then can bisect (horizontally
  and verically) the rectangle $R$ into four smaller, equally large rectangles:
  $S,T,U,V$, so that $I(R)=I(S)+I(T)+I(U)+I(V)$ (since the integrals along edges
  inside $R$ cancel out). We can then choose a small rectangle, say $R_1=S$,
  such that 
  \[|I(R)|\leq 4|I(R_1)|.\]
  Following this proceadure we can construct a sequence of rectangles $R\supset
  R_1 \supset R_2 \cdots$, where each rectangle is the quarter of its
  predecessor, hence 
  \[|I(R)| \leq 4^n |I(R_n)|.\]
  Let $(s_n,t_n)$ be the lower left corner of the rectangle $R_n$. The sequence
  $s_1,s_2,\cdots$ is increasing and bounded above, so it converges to a limit
  $s$, and similarly $t_n$ converges to $t$. Let $w=\Gamma(s,t)$, the image of
  this limit point $(s,t)$ under the continuous function $\Gamma$. Then $w\in
  \Gamma(R_n)$ for every $n$ (without the image this is hard to see, but imagine
    bisecting the rectangle in four sides recursively, and observe how the limit
  point must be in every bisection in the sequence $R_1,R_2,\cdots$ ).

  We now use the fact that $f$ is differentiable. For $z\in \Omega\setminus \{w\}$,
  let 
  \[q(z) = \frac{f(z)-f(w)}{z-w} - f'(w),\]
  And define $q(w)=0$. Then $q$ is continuous (by definition of derivative of
  $f$) on $\Omega$, and 
  \[f(z)=f(w)+f'(w)(z-w) + q(z)(z-w)\]
  For all $z\in\Omega$. Since the integral of a polynomial along a closed curve
  is $0$ (we showed $\oint z^n dz=0$ when $n\neq -1$), we have 
  \[I(R_n)= \oint_{\Gamma|\partial R_n} q(z)(z-w) dz.\]
  What we now want to do is to show how this integral vanishes as $n$ tends to
  infinity. Note that the side lengths of $R_n$ are related to those of $R$ by
  $2^{-n}$ (since bisecting at every iteration, we're halving the sides every
  iteration). Moreover, we can choose $\Gamma:R\to\Omega$ s.t. lengths in $R$
  get scaled by a finite amount when mappend to $\Omega$ under $\Gamma$. I.e.
  there exists constants $K,L>0$ s.t. 
  \begin{enumerate}
    \item The length of $\gamma(R_n)=\Gamma|\partial R_n \leq 2^{-n}K$,
    \item $|z-w|\leq 2^{-n}L$ for every $z\in \Gamma|\partial R_n$,
    \item Since $q(z)\to 0$ as $z\to w$ (by definition), the function $q$ must
      be bounded, i.e. $|q(z)|\leq M_n \forall z\in \Gamma|\partial R_n$ with
      $M_n\to 0$ as $n\to\infty$.
  \end{enumerate}
  Therefore,
  \[|I(R)|\leq |4^n I(R_n)| = |4^n \oint_{\Gamma|\partial R_n} q(z)(z-w) dz|\]
  Recall that if $\gamma$ is of finite length $L$ and $|f(z)|\leq M$ for all
  $z$, then $|\int_{\gamma} f(z)dz| \leq ML$. Hence the above reduces to 
  \[|I(R)|\leq 4^n M_n 2^{-n}L 2^{-n}K = M_nLK \to 0,\]
  As required.
\end{proof}

Next, we use this result to show how, in some circumstances, the integrals of a
function along different paths are equal (i.e. how the above result implies
homotopy).
\begin{definition}
  Let $\gamma,\delta:[a,b]\to X$ be two closed paths with the same domain
  $[a,b]$ and the same codomain $X$. Then $\gamma,\delta$ are homotopic as
  closed paths in $X$ if there exists a continuous functions
  $\Gamma:[0,1]\times [a,b]\to X$ such that
  \[\Gamma(0,t)= \gamma(t), (t\in[a,b])\]
  \[\Gamma(1,t)= \delta(t), (t\in[a,b])\]
  \[\Gamma(s,a)= \Gamma(s,b), (s\in[0,1])\]
  The function $\Gamma$ is called a \emph{homotopy} from $\gamma$ to $\delta$.
  \label{def:homotopy}
\end{definition}
Under this definition, the function $\Gamma$ can be seen as a continuous family
of  closed paths, each one for each point $s$ in $[0,1]$, starting with $\gamma$
and finishing with $\delta$. With this definition, we're ready to show the
circumstances in which the integrals of functions along different paths are
equal.

\begin{theorem}[Homotopy invariance]
   Let $f:\Omega\to\CC$ be a differentiable function on an open set $\Omega$ in $\CC$ and
   let $\gamma$ and $\delta$ be closed paths in $\Omega$ of finite length which
   are homotopic as closed paths in $X$. Then we have
   \[\oint_{\gamma} f(z)dz = \oint_{\delta} f(z)dz.\]
  \label{thm:homotopyInvariance}
\end{theorem}
\begin{proof}
  We can find a homotopy $\Gamma: [0,1]\times [a,b]\to\Omega$ whose restriction
  to the boundary $\partial X$ of its domain is of finite length. The integral of f along
  this boundary (in the appropriate direction) is then
  \[\oint_{\Gamma|\partial X}f(z)dz = \oint_{\gamma}f(z)dz -
  \oint_{\delta}f(z)dz.\]
  By Theorem \ref{thm:CauchyIntegrationRectangles}, the integral along the image
  of the boundary of a rectangle of a continuous function is $0$ (where $f$ is
  continuous inside the image of that rectangle), hence we have
  \[\oint_{\gamma}f(z)dz - \oint_{\delta}f(z)dz = 0,\]
  As required.
\end{proof}
We can then use this to generalise Theorem
\ref{thm:CauchyIntegrationRectangles}. 
\begin{theorem}[Cauchy's theorem for null-homotopic paths]
  Let $f:\Omega\to\CC$ be a differentiable function on an open set $\Omega$ in $\CC$ and
  let $\gamma$ be a closed path in $\Omega$ of finite length which is homotopic to a
  constant path (as a closed path) in $\Omega$. That is, $\gamma$ is
  null-homotopic in $\Omega$. Then,
  \[\int_{\gamma} f(z)dz =0.\]
  \label{thm:CauchyNullHomotopic}
\end{theorem}
\begin{proof}
  This follows from Theorem \ref{thm:homotopyInvariance} about homotopy
  invariance, since the integral in a constant path (essentially, a point in
  $\CC$) is obviously $0$.
\end{proof}
As a special case, we can apply this result to paths that do not cross
themselves, called simple closed.


















