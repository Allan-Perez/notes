\section{Lecture 13 - 3 Mar 2022 - Laurent Series}
Let $\Omega\subset$ be open and $f:\Omega\to\CC$ be holomorphic. Pick
$c\in\Omega$ then we can find a disk $D\subset\Omega$ at $c$ such that for any
$z\in D_c$ we have
\[f(z)=\sum_{n=0}^{\infty} a_n (z-c)^n\]
With coefficients given by
\[a_n=\frac{1}{2\pi i} \oint_{\gamma}\frac{f(\zeta)}{(\zeta-z)^{n+1}}d\zeta.\]
 Where $\gamma$ is a circle centered at $c$ and oriented anticlockwise. This is
 the case for a holomorphic function with simply connected domain. However, what
 happens when the domain is not simply connected (e.g. when there's a
 singularity). The following generalises the above statement (which is really
 just Taylor's series with Cauchy's integral formula) to cases where the domain is punctured.
 \begin{theorem}
   Let $f:\Omega\setminus\{c\}\to\CC$ be holomorphic with $\Omega\subset\CC$
   open and $c\in \Omega$. Then there exists an open disk $D$ with center at $c$
   and a circle $\gamma$ traced anticlockwise s.t.
   \[f(z)=\sum_{n=0}^{\infty} a_n(z-c)^n + \sum_{n=1}^{\infty} a_{-n}
   (z-c)^{-n}\]
   With 
   \[a_n=\frac{1}{2\pi i} \oint_{\gamma}\frac{f(\zeta)}{(\zeta-z)^{n+1}}d\zeta.\]
  For all $z\in\Omega\setminus\{z\}$. The latter expansion of $f$ is called its
  Laurent series at $c$.
   \label{thm:laurentSeries}
 \end{theorem}
 \begin{proof}
   Decompose the Laurent series into two parts, 
   \[S^+=\sum_{n=0}^{\infty} a_n(z-c)^n, \quad S^-= \sum_{n=1}^{\infty} a_{-n}
   (z-c)^{-n}\]
   We claim $S^{+}+S^{-}=f(z)$. To evaluate $S^{+}$, we pick a circle $\gamma^+$
   with center $c$ s.t. $z$ lies inside $\gamma^+$. As a consquence, as we saw
   before, the series $\sum_{n=0}^{\infty}
   f(\zeta)\frac{(z-c)^n}{(\zeta-c)^{n+1}}$ converges absolutely. Hence, 
   \[\frac{1}{2\pi i} \oint_{\gamma^+} \sum_{n=0}^{\infty} f(\zeta)
   \frac{(z-c)^{n}}{(\zeta -c)^{n+1}} d\zeta = \sum_{n=0}^{\infty}
   \frac{1}{2\pi i} \oint_{\gamma^+}
   f(\zeta)\frac{(z-c)^{n}}{(\zeta-c)^{n+1}}\zeta = S^+.\]

   To evaluate $S^{-}$ we choose a circle $\gamma^-$, with center at $c$,
   with $z$ outside $\gamma^-$, and with clockwise direction. In this case we
   clearly have $|\frac{\zeta-c}{z-c}|<1$, hence the following equality is
   permitted since the geometric series converges absolutely,
   \[\frac{1}{2\pi i}\oint_{\gamma^-}\sum_{n=0}^{\infty} f(\zeta)
     \frac{(\zeta-c)^{n}}{(z-c)^{n+1}}d\zeta = \sum_{n=0}^{\infty}
   \frac{1}{2\pi i} \oint_{\gamma^-}
   f(\zeta)\frac{(\zeta-c)^n}{(z-c)^{n+1}}d\zeta = S^-.\]
   With these equalities we obtain the following values (simply by computing the
   limit of each geometric series),
   \[S^{+} = \frac{1}{2\pi i} \oint_{\gamma^+} \frac{f(\zeta)}{\zeta-z}d\zeta,\]
   \[S^{-} = \frac{-1}{2\pi i} \oint_{\gamma^-} \frac{f(\zeta)}{\zeta-z}d\zeta,\]
   Recall that changing the sign of a contour integral is equivalent to changing
   the direction of travel. Let $\sigma$ be a closed path in $\Omega$ consisting
   of the circle $\gamma^+$ traced anticlockwise, a path $\lambda$ from a point
   on $\gamma^+$ to a point on $\gamma^-$, the circle $\gamma^-$ traced
   clockwise, and the path $\lambda$ traced in reverse. Hence we observe that 
   \[S^+ + S^- = \frac{1}{2\pi i} \oint_{\sigma}\frac{f(\zeta)}{\zeta-z} d\zeta.\]
   This path is homotopic to a simple closed path $\tau$ that surrounds $z$  but
   not $c$. Hence by using this path, we use Cauchy and it follows that
   $f(z)=S^+ + S^-$, as required.
 \end{proof}
