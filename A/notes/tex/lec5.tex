\section{Lecture 5 - 29 Sep 2021}
\subsection{Counting groups and Lagrange}
\begin{definition}
  Let $G$ be a group and $H$ a subgroup. The set of left cosets of $H$ in $G$ is denoted
  by $G/H$, also called the set of equivalence classes generated by the equivalence
  relation defined in Corollary \ref{leftCosetsEqRel} . Similarly the set of right cosets of
  $H$ in $G$ is denoted by $H\setminus G$.
  \label{cosets}
\end{definition}
Note that everything we said about left cosets apply symmetrically to right cosets, but
the partitions of a given subgroup $H$ may be different between left and right cosets.

\begin{corollary}
  Let $G$ be a finite group, and let $H$ be a subgroup. Then $|H|$ divides $|G|$
\end{corollary}

\begin{definition}
  The number of (left) cosets of $H$ in $G$ is called the index of $H$ in $G$, written
  $[G:H]$.
\end{definition}
The index may be infinite.

\begin{corollary}
  Let $G$ be a finite group and $H$ be a subgroup. Then we have that $|G|=|H| \cdot [G:H]$.
  \label{lagrange}
\end{corollary}
\begin{proof}
  By Theorem \ref{cosetsCardinality}, we have that $G$ is the union of cosets, which are
  disjoint. I.e. $G=\bigcup_{g\in G} gH\implies |G|=|gH|[G:H]=|H|[G:H]$.
\end{proof}

\begin{theorem}
  The number of left cosets of $H$ in $G$ equals the number of right cosets. Note that $G$
  may be an infinite-order group.
\end{theorem}
\begin{proof}
  Define a bijection $G/H\to H\setminus G$ as $gH\mapsto Hg^{-1}$. To show injectivity,
  suppose $Hg^{-1}=Hg'^{-1}$ for some $g,g'\in G$. Then $\exists h\in H$ s.t.
  $hg^{-1} = g'^{-1}$, so $g' = gh^{-1}\in gH \implies g'H=gH$. Surjectivity follows from
  the fact that every element in $G$ has an inverse, so it's not possible that an element
  of the image of the defined mapping is not mapped.
\end{proof}
Note that we may have infinite-order groups but with finite index (finite number of
cosets).

\begin{corollary}
  Let $G$ be a finite group, and let $g\in G$. Then the order of $g$ divides $|G|$.
  \label{lagrangeConsequence}
\end{corollary}
\begin{proof}
  Let $H=<g>$ so that $H<G$. By Lagrange we have $|G|=|H|[G:H]=|g|k$ for $k=[G:H]$.
\end{proof}

\begin{example}
  A particular example where Lagrange theorem is useful is the following. Let
  $n\in\NN$. The set $\left\{ i\in \left\{ 1,2,\cdots,n-1 \right\} : gcd(i,n)=1 \right\}$
  forms a group under multiplication mod $n$. This group is denoted by
  $(\ZZ/n\ZZ)^{\times}$. Note that if $n$ is prime, then the group has order $n-1$. Let us
  work out the order of $3\in (\ZZ/7\ZZ)^{\times}$. By Lagrange, we know that $|3|$ must
  divide $|G|=6$, so $|3|$ can be either $2,3,$ or $6$. It's not hard to see that
  $3^2=9=2\mod 7$ and $3^3=2*3\mod 7=6\mod 7$, so we must have $|3|$ has indeed order $6$. Since the group
  is order $6$ and we found an element of order $6$, we also discovered that the group is
  cyclic ($3$ is a generator).
\end{example}
