\section{Lecture 25 - 17 Nov 2021 }
\subsection{Prime ideals in polynomial rings}
\begin{corollary}
  Let $F$ be a field and let $f\in F[X]$ be a non-zero polynomial with $n=\deg f$. Then
  $f$ has at most $n$ roots.
  \label{<+label+>}
\end{corollary}
\begin{proof}
  Applying the previous corollay inductively. If $a_1,\cdots, a_n$ are distinct zeros of
  $f$, then by the previous corollary we have $f(x)=\prod_{i=1}^n (x-a_i) h(x)$, and since
  the linear polynomial term already will have, after expanding, degree $n$, then the
  product with $h$ will have degree at least $n$.
\end{proof}

\begin{definition}
  Let $f\in F[X]$ be a polynomial over the field $F$ with $\deg f>0$. Then $f$ is
  \emph{irreducible} if whenever $g,h\in F[X]$ are such that $f=gh$, one has either $\deg
  g=0$ or $\deg h=0$. If $f$ is not irreducible, then it's called \emph{reducible}.
\end{definition}
\begin{remark}
  The notion of irreducability is relative to the field you're working on.
\end{remark}
\begin{example}
  Consider $f(x)=X^2+1$. This is irreducible in $\RR[X]$. However, in $\CC[X]$, we can
  write $f(x)=(X-i)(X+i)$, hence $f$ is reducible in $\CC[X]$.
\end{example}

\begin{theorem}
  Let $F$ be a field and let $I\subseteq F[X]$ be an ideal. Then there exists $f\in
  F[X]$ s.t. $I=(f)=\{gf : g\in F[X]\}$.
  \label{thm:idealPrinciplas}
\end{theorem}
\begin{proof}
  If $I=\{0\}$, then $I=(0)$. Next, consider $g\in I\setminus\{0\}$ be s.t. $\forall g'\in
  I\setminus\{0\}$, $\deg g \leq \deg g'$. Then since $I$ is an ideal,  $\forall h\in
  I\exists f\in F[X]$ s.t.  $gf=h$. In order words, 
  \[I= \{ 0, g, f_1g, f_2g, \cdots f_i\in F[X] \}\]
  Since $I$ is an ideal, we have that for every $h=f_k g \in I$ and for every $l\in F[X]$
  we have $lh\in I$, so $lfg \in I$. In other words, $I=\{fg | \forall f\in F[X]\}$, or
  $I=(g)$.
\end{proof}
\begin{proof}[Bartel's proof]
  If $I=\{0\}$, then $I=(0)$. Consider $f\in I\setminus \{0\}$ be of the smallest degree
  that is non-zero. We calim $I=(f)$. Let $g\in I$ and note that by Theorem
  \ref{thm:divisionReminderPolynomial} we have that there exists unique $q,r\in F[X]$ s.t.
  $g=qf+r$ and $r=0$ or $\deg r < \deg f$. Since we're considering the ideal $I$, it
  follos that $qf\in I$ and $g\in I$ by assumption, hence $r\in I$ and $r=g-qf$. But since
  we have $\deg f$ is the minimal, it follows that $r=0$ (because $0<\deg r<\deg f$ would
  contradict that $f$ was the smallest non-zero degree). Since $g$ was arbitrary, it
  follows that $I=(f)$.
\end{proof}
\begin{remark}
  This theorem does not hold if $F$ is not a field.
\end{remark}

\begin{example}
  Let $I$ be the ideal of $\QQ[X]$ generated by $(X^2+X)$ and $(X^4+X^3+X)$, 
  \[I=\{f(X^2+X) + g(X^4+X^3+X) : f,g\in \QQ[X]\}\]
  By the above theorem, we have that 
  \[I=(X)\]
\end{example}
