\section{Lecture 6 - 4 Oct 2021}
This lecture we will start the revision on normal subgroups and quotients.
Recall from last year that, if $V$ is a vector space and $U$ is a subspace, then the set
of cosets $\left\{ v+ U: v\in V \right\}$ is a vector space in its own right. For example,
the addition of cosets is defined by $(v+U)+(v'+U)=(v+v')+U$. The lecturer gave as an
intuitive example the vector space $\RR^2$, where we take a line throught the origin as a
subspace, and then show how adding some vector form $\RR^2$ just shifts the line around
the plane (no longer through the origin, so not a subspace), and adding these two lines we
add the two elements used to create that line ($v,v'$), find the new vector ($v+v'$), and
create a new line.

In order to be sure that this addition of cosets works, we need to show that it's well
defined (i.e. it preserves under addition of different representative of the same coset).


Let us try to generalise this idea of reconstructing the parent structure from quotients,
but to any group. Let $G$ be a group and $H$ a subgroup of $G$. We would like the set of
left cosets of $H$ in $G$ to \emph{inherit} the group structure. We could try to define
for $x,y\in G$, $xH yH=(xy)H$. We need to check that this is well defined, as above. Let
$e,h\in H$ denote the identity element and some element of $H$, respectively. Let $y\in G$
denote some element not in $H$. We want that $(eH)(yH)=yH = (hH)(yH)=(hy)H \iff
y^{-1}hy\in H$. 

Hence, a necessary and sufficient condition for multiplication of cosets
to be well defined is that, for every $g\in G$ and $h\in H$ we get $ghg^{-1}\in H$.
Another way of stating this condition is,
\[\forall g\in G, gHg^{-1}\subseteq H \iff H\subseteq g^{-1}H g \iff H=g^{-1}H g\]

\begin{definition}
  Let $G$ be a group. A subgroup $H$ of $G$ is called normal, written $H\triangleleft G$
  or $H\trianglelefteq G$, if $\forall g\in G$ we have $gHg^{-1}=H$.
  \label{normalSubgroup}
\end{definition}

Hence for the set of left (or right) cosets of $H<G$ to have group structure with the
above operation, we require $H$ to be normal in $G$.

\begin{example}
  Consider the group $S_3$. We claim the subgroup generated by $(1,2,3)$ is normal.
  Indeed, it is the group consisting of the identity and all 3-cycles in $S_3$. Since
  conjugation preserves the cycle type of a permutation, the claim follows (since any
  conjugation will be a 3-cycle, which is contained in the subgroup).

  By contrast, the subgroup generated by $(1,2)$ is not normal. Get for instance
  \[(1,3)(1,2)(1,3)=(2,3)\not\in<(1,2)>\]

\end{example}

Note: Conjugation preserves the order of an element in any arbitrary group (we prove this
last year, since the conjugating elements get cancelled out). However, the claim above is
stronger: conjugating in a symmetric group element preserves the \emph{cycle type}, i.e.
if an element is 3-4-2-cycle (e.g. $(1,2,3)(4,5,6,7)(8,9)$), conjugating it will yield
again an element of 3-4-2-cycle.

\begin{example}
  Recall that we can write a permutation as a product of transpositions (2-cycle
  permutations), and the \emph{parity} of the number of transpositions is invariant
  (doesn't depend how you write the product). So a permutation is even if they can be
  written as an even number of transpositions, and it's odd otherwise. Hence we define the
  sign of a permutation to be $+1$ or $-1$ depending on whether it's even or odd,
  respectively.

  Let $n\in\NN$ and let $A_n\subset S_n$ be the set of even permutations. This is a normal
  subgroup, since for any $\sigma, \tau\in S_n$ we have
  \[\sign [\sigma \tau \sigma^{-1}] =
  \sign[\sigma]\sign[\tau]\sign[\sigma]=\sign[\tau]\]
\end{example}

\begin{example}
  Let $D_{2n}$ be the dihedral group of order $2n$, and let $H$ be the subgroup of $n$
  rotations. Then $H$ is normal. On the other hand, the subgroup generated by a refletion
  is not normal.

  Another example is any abelian group, since every subgroup is normal.

  Finally, for a group $G$, any trivial subgroups of the group itself are always normal,
  no matter the group.
\end{example}

Once we've demonstrated well-definedness and provided some examples of the criteria, we
have a method for creating a group structure out of the quotients of the group. 

\begin{definition}
  Let $G$ be a group, and $N$ be a normal subgroup. The set of left cosets $G/N$ together
  with the binary operation $(gN)(hN)=(gh)N$ for $g,h\in G$ is called the \emph{quotient
  group} or \emph{factor group} of $G$ by $N$.
  \label{quotientGroup}
\end{definition}
Note that by the definition of normal subgroups, we have that normal subgroup's left
cosets equal that subgroup's right cosets, since $gNg^{-1}=N\iff gN=Ng$.
