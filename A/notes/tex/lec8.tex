\section{Lecture 8 - 8 Oct 2021}
\subsection{Group Homomorphisms, Types and Facts}
On group homomorphisms. 
\begin{definition}
  Let $G,G'$ be groups. A group homomorphism from $G$ to $G'$ is a function $\phi:G\to G'$
  s.t. for all $g,h\in G$ one has $\phi(gh)=\phi(g) \phi(h)$.
  \label{groupHomomorphism}
\end{definition}
I.e. with the group homomorphism, the group operation in $G$ is the same as the group
operation in $G'$.

\begin{example}
  For any two groups, there's always at least one group, being the trivial homomorphism
  (sending every element to the identity element).
\end{example}

\begin{example}
  Write $\RR^x$ as the non-zero reals as a group under multiplication. For every $n\in\NN$
  we have the group homomorphism
  \[ \phi: \GL_n \RR \to \RR^x:X\mapsto \det X\]
  Where $\phi(XY)=\phi(X)\phi(Y)$. This works for any field, let it be $\QQ,\CC$.
\end{example}

\begin{example}
  For every $n\in\NN$, the function $S_n\to \left\{ \pm 1 \right\}; \sigma\mapsto
  \sign\sigma$ is a group homomorphism. I.e. $\sign (\sigma\tau) = \sign \sigma \sign
  \tau$.
\end{example}

\begin{example}
  Let $G$ be a group and $H$ be a normal subgroup. The quotient map $G\to G/H : g\to gH$
  is a surjective group homomorphism. Note that it's surjective because every element is
  mapped, and homomorphism follows from the defining group operation of the quotient
  group. In particular, for every $n\in\NN$ there is a surjective homomorphism $\ZZ\to
  n\ZZ$, $k\mapsto k+n\ZZ$.
\end{example}

Recall that we established multiplicative notation for non-abelian group operations and
additive notation for abelian group operations. What if a group homomorphism sends from
one non-abelian to an abelian group? Then the notation in the homomorphism has to be
changed accordingly.
\begin{example}
  Te set $\RR_{>0}$ is a group under multiplication. The homomorphism $\RR_{>0}\to
  \RR:x\mapsto \log x$. We have $\log xy = \log x + \log y$. Similarly we have a
  homomorphism $\RR\to\RR_{>0} : x\mapsto e^{x}$.
\end{example}

The following theorem is foundational. The first two points establlish that groups form a
category. 
\begin{theorem} [Groups form a category]
  We have the following
  \begin{enumerate}
    \item Let $G,G',G''$ be groups and $\phi:G\to G'$ and $\phi':G'\to G''$ be group
      homomorphisms. Then the composition $\phi\circ\phi':G\to G''$ form also a group
      homomorphism.
    \item Let $G$ be a group. The identity map $G\to G,g\mapsto g$ is a group
      homomorphism.
    \item Let $G,G'$ be groups, and define a bijective group homomorphism $\phi:G\to G'$.
      Then the inverse function $\phi^{-1}:G'\to G$ is also a group homomorphism.
  \end{enumerate}
  \label{groupsCategories}
\end{theorem}
\begin{proof}
  \todo{Check the work. Improve the labeling of variables. Finish the last part}
  Let $g,h\in G, g',h'\in G', g'',h''\in G''$ throughout.
  \begin{enumerate}
    \item We have $\phi\circ\phi'(gh)= \phi'(\phi(gh))=
      \phi'(\phi(g)\phi(h))=\phi'(g'h')=g''h''=\phi\circ\phi'g \phi\circ\phi'h $. As
      reqruied.
    \item We have $\phi gh=1 = 1\cdot 1 = \phi g \phi h$ for any $g,h\in G$. As required.
    \item We claim $\phi^{-1}(g'h')= \phi^{-1}(g')\phi^{-1}(h')$.  

  \end{enumerate}
\end{proof}

\begin{theorem}
  Let $\phi:G\to G'$ be a group homomorphism. Then
  \begin{enumerate}
    \item $\phi(1_G)=1_{G'}$
    \item for every $g\in G$, $\phi(g^{-1})= \phi(g)^{-1}$
  \end{enumerate}
  \label{homIdInv}
\end{theorem}
\begin{proof}
  \begin{enumerate}
    \item Let $\phi(1_G)=g'\in G'$ that is not the identity. Then $\phi(1_G)=\phi(1_G
      1_G)=\phi(1_G)\phi(1_G)=g'g'=g'^2=g'=\phi(1_G1_G1_G)=g'^3$. This is a contradiction.
    \item We have $\phi(g^{-1}g)=\phi(g^{-1})\phi(g)= 1_G'$ by the above. We also must
      have $\phi(g)^{-1}\phi(g)=\phi(g)\phi(g)^{-1}=1_G'$ since $G'$ is also a group.
      Hence we have $\phi(g)^{-1}\phi(g)=\phi(g^{-1})\phi(g) \iff
      \phi(g)^{-1}=\phi(g^{-1})$
  \end{enumerate}
\end{proof}


\begin{definition}[Morphism Zoo]
  \begin{enumerate}
    \item A group \emph{isomorphism} is a grou homomorphism $\phi:G\to G'$ that has a
      2-sided inverse. This is, $\phi\circ\phi^{-1}=1_{G'},\phi^{-1}\circ\phi=1_G$. If
      there exists a group isomorphism between groups $G,G'$, we say these groups are
      isomorphic and write $G\cong G'$. By the previous results (Theorems
      \ref{groupsCategories} \ref{homIdInv}) we have that being isomorphic is an
      equivalence relation. Two isomorphic groups are structurally the same.

    \item A group automorphism is a group isomorphism to itself.
    \item A group endomorphis is a group homomorphism to itself. E.g. the trivial
      homomorphism to itself is an endomorphism for non-trivial groups.
  \end{enumerate}
  \label{morphismZoo}
\end{definition}

The following is what in last year we were taught to be the definition of an isomorphism,
but it really is a theorem.
\begin{theorem}
  Let $G,G'$ be groups and $\phi:G\to G'$ be group homomorphism. Then $\phi$ is an
  isomorphism if and only if $\phi$ is bijective.
  \label{isomorphismBijective}
\end{theorem}
\begin{proof}
  If the homomorphism is an isomorphism, we have that it must have a two-sided inverse,
  and it follows that $\phi$ is bijective. 

  If $\phi$ is bijective, we claim that $\phi^{-1}$ exists and that it is a group
  homomorphism. This follows from theorem \ref{groupsCategories}.
\end{proof}
