\section{Lecture 7 - 6 Oct 2021}
In this lecture we will focus on examples of quotient groups.

\begin{example}
  Let $G=S_n$ and $N=A_n$. There are exactly two left cosets of $A_n$ in $S_n$: $1A_n,
  (1,2)A_n$, the latter consisting of all odd permutations. Hence the quotient $S_n/A_n$
  is cyclic of order $2$.
\end{example}

\begin{example}
  Consider the group $S_4$. The subgroup $V_4$ consisting of all 2-cycles, plus the
  identity, is normal. The set $X=\left\{ e,(1,2),(1,3),(2,3),(1,2,3),(1,3,2) \right\}$ is
  a \emph{full set of left coset representatives of $V_4$ in $S_4$}, i.e. every coset of
  $V_4$ in $S_4$ contains exactly one element in $X$. This happens to be a subgroup of
  $S_4$, actually it's $S_3$. Hence we immediately identify the quotient group $S_4/V_4$
  is isomorphic to $S_3$.
\end{example}

\begin{example}
  Consider the group $\ZZ$ and the normal subgroup $n\ZZ$ for $n\in\NN$. The quotient
  $\ZZ/n\ZZ=\left\{ 0+n\ZZ, 1+n\ZZ, \cdots, n-1+n\ZZ \right\}$ is cyclic of order $n$,
  generated by $1+n\ZZ$. This quotient group may be written as $C_n$.
\end{example}

\begin{definition}
  Let $(H,*),(K,\cdot)$ be groups. We define their \emph{direct product} as the group
  with underlying set $H\times K = \left\{ (h,k) : h\in H, k\in K \right\}$ with operation
  of point-wise multiplication
  \[(h,k)(h',k')=(h*h',k\cdot k')\]
  \label{directProduct}
\end{definition}
\todo{Verify this defines a group}

We then have that the subset $1\times K=\left\{ (1,k) :k\in K \right\}$ of the direct
product $H\times K$ is a normal subgroup, and a full set of cosets representatives is
$\left\{ (h,1) : h\in H \right\}$, since
\[(h,k) 1\times K = \left\{ (h,kk') :k'\in K \right\}\]
\[= \left\{ (h, k'') : k''\in K \right\}\]
As it turns out, the second entry of the element we're multiplying with to generate a
coset doesn't matter (given $h$, any $k$ will generate the same coset). Hence $(H\times
K)/(1\times K)$ can be identified back with $H$.
