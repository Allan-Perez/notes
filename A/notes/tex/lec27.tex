\section{Lecture 27 - 22 Nov 2021}
\subsection{Intermission: Classifying groups of order 21}
\begin{example}
  Let $G$ be a group of order 21. How many possibilities are there for $G$.
\end{example}
\begin{proof}
  Note that $21=7\cdot 3$, and by Cauchy, there are elements of $G$, say $g,h\in G$, with
  order $7$ and $3$, say $g^7=e=h^3$. Let $K=\langle g\rangle$ and $H=\langle h\rangle$.
  Note that $K\cap H=\{1\}$, since by Lagrange we have that $|K\cap H|$ divides $|K|$ and
  $|H|$, and $\gcd(|H|,|K|)=1$, and the intersection is required to be a group itself.
  Moreover, note that $K\trianglelefteq G$ since $[G:K]$ being the smallest prime divisor
  of $G$ implies $K\trianglelefteq G$.

  Next, let $h\in H$ and consider the automorphism $\phi_h:K\to K:g^i\mapsto hg^ih^{-1}$
  (it's rather easy to see that the map is well defined and surjective, by normality, it's
  a homomorphism by the general argument of conjugation as a homomorphism, and it is an
  isomorphism because the inverse is $\phi_{h^{-1}}$). Suppose that $hgh^{-1}=g^r$, and note
  that the choice for $r$ will completely determine the automorphism (since $g$ generates
  $K$). Note that $\phi_h^{j}(g)=h^jgh^{-j}=g^{r^j}$ and we require $j=3\implies h^3=e$
  hence we require $g^{r^j}=g^1$. Hence we require $r^3\cong 1 \mod 7$, which are obiously
  $r=1$, $r=2$, $r=4$. 

  An alternate point of view is that, the function $H\to \Aut K:x\mapsto \phi_x$, where
  $\Aut K$ is a group under composition and the identity is the identity automorphism, and
  note that there are $|K|-1=6$ possible such automorphisms.  Moreover, we have $K$ is
  cyclic of order $7$, i.e. $\Aut K \cong (\ZZ/7\ZZ)^{\times}$, its automorphism group is
  group under composition. More generally we have that $\Aut(\ZZ/n\ZZ)\cong
  (\ZZ/n\ZZ)^{\times}$.  In fact, note that when $n$ is prime, the unit groups before is
  cyclic of order $n-1$. Note also that for a cyclic group of order $n$, every elemenet of an
  order dividing $n$ generates exactly one subgroup of that order. So $\Aut K$ being
  a cylic of order $6$ we have that there is exactly one subgroup of order $3$.

  Hence we have the following possibilities. 
  \emph{Case 1: $\phi_h(g)=g$} -- Then $gh=hg$. Note that the semidirect product using
  this conjugation rule gives $HK=G$ since they intersect trivially and the product of
  their order is the order of the group, and the fact that the elements commute gives
  raise to the fact that the group is cylic with generator $gh$.
  \emph{Case 2: $\phi_h(g)=g^2$} -- Non-commuting where the semidirect product of $H$ and
  $K$ using this conjugation rule gives $K\rtimes H\cong C_7\rtimes C_3$.
  \emph{Case 3: $\phi_h(g)=g^4$} -- Note that this hs the case for the previous
  $\phi_h(g)=g^2$, but $\phi_h^2(g)=\phi_{h^{-1}}=g^4$ (since $H$ has order 3).
\end{proof}<++>
