\section{Lecture 2 - 22 Sep 2021}
 \begin{definition}
   A subgroup of a group G, $H\subset G$, is a subset of $G$ with 
   \begin{itemize}
       \ii It contains the $G$ group identity element.
       \ii For all $a,b\in H$ we have $ab\in H$
       \ii for any $a\in H$ we have $a^{-1}\in H$.
   \end{itemize}
   \label{subgroup}
 \end{definition}
Basically, $H$ is a subset that is a group on its own, with the same identity and same
group operation. We write $H<G$ or $H\leq G$ to denote a subgroup.

\begin{proposition}[Test for subgroup]
  Let $H$ be a subset of the group $G$. Then $H$is a subgroup iff the following hold:
  \begin{enumerate}
    \item $H\neq\emptyset$
    \item If $x,y\in H$ then $x^{-1}y\in H$.
  \end{enumerate}
  \label{subgroupTest}
\end{proposition}
\begin{proof}
  The first holds since $e\in H$. 

  ($\implies$) Since we have that $H$ is a subgroup, then we must have $x\in H \implies
  x^{-1}\in H$ and $x^{-1}y\in H$ for some $y\in H$.

  ($\Leftarrow$) We have that $H$ is non-empty and that for any $x,y\in H$ we have
  $x^{-1}y\in H$. We claim that $H$ contains the identity, is closed, associative, and
  every element has an inverse. Note that the identity is in $H$, since when $x=y$, the
  result follows. Also, note that it's closed, since $x^{-1}y=h\implies
  h^{-1}=y^{-1}x\in H$. This also implies that every element has an inverse. Finally, to
  show associativity, we claim $a,b,c\in H$ we have $(ab)c=a(bc)$. Note that given the
  above, we have $(ab)c=a(bc)\iff a'(ab)c=a'a(bc)=bc=(a'a)bc \iff a'(ab)c=(a'a)bc= a'abc$.
\end{proof}

\begin{definition}
  A group $G$ is called cyclic if $\exists g\in G$ such that $G=\left\{ g^n : n\in\ZZ
  \right\}$. If $G$ is cyclic, then an element $g$ as above is called a generator, and we
  say $G$ is generated by $g$, $G=<g>$.
  \label{cyclicGroup}
\end{definition}

\begin{theorem}
  Every cyclic group is abelian.
\end{theorem}
\begin{proof}
  $g^n g^m = g^{n+m} = g^{m+n} = g^m g^n$
\end{proof}

\begin{theorem}
  All subgroups of a cyclic group are cyclic.
\end{theorem}
\begin{proof}
  Let $H<G$ where $G$ is a cyclic subgroup. We claim $\exists h\in H$ s.t. $H=\left\{
  h^n | n\in\ZZ  \right\}$. When $H=\left\{ 1_G \right\}$, the proof is trivial. If
  $\exists n\in\ZZ$ s.t. $g^n\in H$, assume such $n$ is the lowest possible, without
  loss of generality. Now assume $g^a\in H$ for some $a=qn+r$ for $q,r\in \ZZ$ and
  $0\leq r < n$. Then $g^a=(g^n)^q g^r$, and we know $g^nq\in H$ since $H$ is supposed
  to be a group on its own. Then we must have $g^r\in H$, but $n$ was the smallest
  possible power of $g$, so $r=0$, i.e. $g^r=e$, and hence $g^a = g^nq$, i.e. $H$ must be
  cyclic.
\end{proof}


\begin{definition}
  The order of an element $g\in G$ of a group $G$, written $|g|$, is the least positive
  integer $n$ s.t. $g^n=e$. If such $n$ doesn't exist, it has infinite order. The order
  of a group $G$, $|G|$, is the cardinallity of the underlying set.
  \label{orderGroup}
\end{definition}

\begin{theorem}
  Let $G$ be a group, and let $g\in G$. The order of $g$ is the same as the order of the
  subgroup $<g> < G$.
\end{theorem}
\begin{proof}
  If $|g|=\infty$, then $g^i=g^k \iff i=k$. If $|g|=m<\infty$, then $<g>=\left\{ 1,g,g^2,
  \cdots, g^{m-1} \right\}$.
\end{proof}

\begin{theorem}
  Let $G$ be a group, and let $g\in G$, and $n\in\ZZ$. Then $g^n=e$ iff $n$ is multiple
  of $|g|$.
\end{theorem}
\begin{proof}
  $n\Big | |g| \implies g^n = e$: We have $n=k|g|$ for some integer $k$. Then
  $g^n=(g^{|g|})^k = e^k = e$.

  $g^n=e \implies n\Big | |g|$: Let $n=|g|m + r$ for integers $m,r$ and $0\leq r < |g|$,
  we have $g^n=g^{|g|m} g^r = e$. We know $g^{|g|}=e$, so $g^n = g^r=e$. However, this can
  be true only if $r=0$, hence $n\Big | |g|$.
\end{proof}

\subsection{Cyclic groups structure}
For a cyclic group of infinite order, the structure of the group resembles
$(\ZZ,+)$, whereas for a cyclic group of finite order $n$, the structure resembles
$(\ZZ_n,
+\mod n)$.
