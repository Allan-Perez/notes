\section{Lecture 1 - 20 Sep 2021}
\subsection{Introduction}
A 2OLDE has the following form,
\[ y'' = f(t,y,y')\]
\[ y' = \frac{dy}{dt}\]
The order of the DE is the degree of the highest derivative. The function $f:\RR^3\to\RR$
defines the DE. The DE is linear if $f$ is linear in arguments $y,y'$.
\begin{definition}
  A second order linear differential equation (2OLDE) with unkown $y$ is an expression of
  the form
  \[a(t)y'' + b(t)y' + c(t)y = h(t)\]
  Where $a,b,c,h:I\subset\RR\to\RR$ are given functions. When $h(t)=0$ the equation is
  said to be homogeneous. Otherwise it's said to be non-homogeneous. If the functions
  $a,b,c$ are constant, the DE is said to be constant coefficient. Otherwise it's said to
  be variable coefficient.

  By dividing the above expression by $a(t)$ we reduce the 2OLDE to
  \[y'' + p(t)y' + q(t)y = f(t)\]
  \label{2olde}
\end{definition}


\begin{theorem}
  If the functions $p,q,f:I\subset\RR\to\RR$ are continuous in their domain, and for
  boundary conditions $y(t_0)=y_0$ and $y'(t_0)=y_1$ for $t_0\in I$ and $y_0,y_1\in\RR$,
  then there exists a unique solution to that initial value problem (IVP).
  
  \label{uniquenessSolutionIVP}
  \begin{proof}
    Omitted. It's an aplication of the contraction mapping theorem from metric spaces.
  \end{proof}
\end{theorem}


\subsection{Homogeneous Eq}
Questions we will pose and solve in the course are the following:
\begin{itemize}
  \ii Why did we add two solutions to get the general one? 
  \ii Why is this solution the general one?
  \ii What does a solution to the non-homogeneous DE look like?
\end{itemize}
To address these questions, operators are useful.

\subsubsection{Operators}
One can think of an operation such as differentiting or integrating as mapping with
function inputs and function outputs. For example, $L$ would make an operation on a
function $y:\RR\to\RR$ as 
\[ L: y\mapsto y''+y \]
Such an $L$ is referred to as functional or operator.

