\section{Lecture 16 - 25 Oct 2021}
\begin{example}
  Solve the eigenvalue problem
  \[y'' + \lambda y=0 ; \quad \lambda\in\RR\]
  Subject to the boundary conditions,
  \[y(0)=0; \quad y(L)=0\]
\end{example}
\begin{proof}
  We consider cases where $\lambda=0, \lambda>0, \lambda<0$.

  When $\lambda =0$, note that we have $y''=0$, so $y=A+Bx$. Since we have $y(0)=y(L)=0= A
  = A+BL$ we have $A=B=0$.
  Therefore, $\lambda=0$ is a trivial solution and so not an eigenvalue and there are no
  nontrivial eigenfunctions.

  When $\lambda>0$. Let $\lambda=\mu^2$, so $y''+\mu^2y=0 \implies y(x)=A\sin(\mu
  x)+B\cos(\mu x)$. We check BVP,
  \[0= y(0) = B\]
  \[0=y(L)= A\sin (\mu L) + 0\]
  Choose $A\neq 0$ because otherwise we have a trivial solution. We then have
  $\mu=\frac{n\pi}{L}$ and it follows $\lambda=\frac{n^2\pi^2}{L^2}$ and 
  \[y(x)= A_n \sin (\frac{n\pi}{L}x)\]
  Note that these solutions describe normal modes of a string

  When we have $\lambda<0$, we let $\lambda=-\mu^2$ for some $\mu>0$ and get the equation
  $y''+\mu^2 y = 0 \implies y(x)=Ae^{\mu x} + Be^{-\mu x}$.
  Recall that $\cosh(x)=\frac{1}{2}(e^{x}+ e^{-x})$ and
  $\sinh(x)=\frac{1}{2}(e^{x}-e^{-x}).$ THen we have 
  \[y(x)=\alpha\cosh(\mu x) + \beta\sinh(\mu x)\]
  With the BCs giving 
  \[\alpha=0\]
  \[\beta\sinh(\mu L)=0\]
  However, note that $\sinh(\mu x)=0 \iff \mu =0$ but we have $\mu>0$, so we have
  $\beta=0$. There're no eigenvalues for the trivial solution $y=0$.

  To conclude, the problem has eigenvalues $\lambda_n=\frac{n^2\pi^2}{L^2}, n\in\NN$ with
  eigenfunctions $\phi_n(x)=A_n\sin(\frac{n\pi x}{L})$.
\end{proof}

\subsection{Inner product spaces}
\begin{definition}
  Let $f,g$ be vectors in a vector space $V$ over the reals. The inner product is a
  apping $\langle  \cdot, \cdot\rangle :V\times V \to \RR $ with the following properties
  \begin{enumerate}
    \item $\langle f,g \rangle =\langle g,f \rangle$
    \item $\langle cf, g \rangle = c\langle f,g \rangle $
    \item $\langle f,g+h \rangle =\langle f,g \rangle +\langle f,h\rangle $
    \item $\langle f,f \rangle \geq 0$ with $\langle f,f \rangle =0 \iff f=0$.
  \end{enumerate}
  \label{def:innerProd}
\end{definition}
Note that a vector space plus an inner product is called an \emph{inner product space}. An
example of an inner product space is $V=\SCC[a,b]$, the space of continuous functions with
the inner product
\[\langle f,g \rangle =\int_a^b f(x)g(x) dx\]
For any $f,g\in \SCC[a,b]$.

\begin{example}
  Show that $\langle f,g \rangle =\int_a^b f(x)g(x)\omega(x) dx$ is an inner product on
  the space $\SCC[a,b]$ where $\omega(x)>0\forall x\in $
\end{example}
\begin{proof}[Solution]
  \[\langle f,g \rangle =\int_a^bf(x)g(x)\omega(x) dx = \int_{a}^b
  g(x)f(x)\omega(x)dx=\langle g,f \rangle \]

  \[\langle cf,g \rangle =\int_a^bcf(x)g(x)\omega(x) dx = c\int_{a}^b
  f(x)g(x)\omega(x)dx=c\langle g,f \rangle \]

  \[\langle f,g+h \rangle =\int_a^bcf(x)(g(x)+h(x))\omega(x) dx = \]
  \[= \int_a^bfg\omega dx + \int_a^b fh\omega dx = \langle f,g \rangle +\langle f,h
  \rangle \]

  \[\langle f,g \rangle = \int_a^b ff\omega dx = \int_a^b f^2 \omega dx \]
  \[= \int_a^b f^2\omega dx \geq 0\]
\end{proof}


\begin{definition}
  Let $f,g$ be vectors in an inner product space. Then they are orthogonal iff
  \[\langle f,g \rangle =0\]
  A set of orthogonal vectors are orthonormal if
  \[\langle f_i,f_j \rangle = 
    \begin{cases}
        1 & i=j \\
        0 & i\neq j
    \end{cases}\]
  The inner product of a vector with itself denotes the norm of a vector, $||f||^2$.
  \label{def:orthonormalSet}
\end{definition}

\begin{lemma}
  If $L$ is a Sturm-Liouville operator with free term $q$ and non-free term $p$, then for
  all $u,v\in\SCC^2[a,b]$ we have 
  \begin{enumerate}
    \item $uL[v]-vL[u] = [p(x)(uv'-vu')]'$ (Lagrange Identity)
    \item $\int_a^b uL[v]-vL[u] = p(x)(uv'-vu')\Big|^{x=b}_{x=a}$ (Green's Identity)
  \end{enumerate}
  \label{<+label+>}
\end{lemma}
\begin{proof}
  By using the definition of $L$ (a Sturm-Liouville operator), we get 
  \[uL[v]-vL[u] = u([pv']'+qv)'- v([pu']'+qu) = u[p'v'+pv'']-v[p'u'+pu'']\]
  Since $uqv-vqu=0$. Rearranging
  \[=p'[uv'-vu'] +p[uv''-u''v] = [p(uv'-vu')']'\]
  Proving Lagrange's identity. Green's identity is obtained by integrating both sides.
\end{proof}
